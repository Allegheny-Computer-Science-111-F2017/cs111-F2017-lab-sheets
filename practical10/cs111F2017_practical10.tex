\documentclass[11pt]{article}

% NOTE: The "Edit" sections are changed for each assignment

% Edit these commands for each assignment

\newcommand{\assignmentduedate}{December 11}
\newcommand{\assignmentassignedate}{December 8}
\newcommand{\assignmentnumber}{Ten}

\newcommand{\labyear}{2017}
\newcommand{\labdueday}{Monday}
\newcommand{\labassignday}{Friday}
\newcommand{\labtime}{9:00 am}

\newcommand{\assigneddate}{Assigned: \labassignday, \assignmentassignedate, \labyear{} at \labtime{}}
\newcommand{\duedate}{Due: \labdueday, \assignmentduedate, \labyear{} at \labtime{}}

% Edit these commands to give the name to the main program

\newcommand{\mainprogram}{\lstinline{InvestigateRecursiveSum}}
\newcommand{\mainprogramsource}{\lstinline{src/main/java/practicalnine/InvestigateRecursiveSum.java}}

\newcommand{\secondprogram}{\lstinline{InvestigateRecursiveSum}}
\newcommand{\secondprogramsource}{\lstinline{src/main/java/practicalnine/InvestigateRecursiveSum.java}}

% Edit this commands to describe key deliverables

\newcommand{\reflection}{\lstinline{writing/recursion.md}}

% Commands to describe key development tasks

% --> Running gatorgrader.sh
\newcommand{\gatorgraderstart}{\command{./gatorgrader.sh --start}}
\newcommand{\gatorgradercheck}{\command{./gatorgrader.sh --check}}

% --> Compiling and running program with gradle
\newcommand{\gradlebuild}{\command{gradle build}}
\newcommand{\gradlerun}{\command{gradle run}}

% Commands to describe key git tasks

% NOTE: Could be improved, problems due to nesting

\newcommand{\gitcommitfile}[1]{\command{git commit #1}}
\newcommand{\gitaddfile}[1]{\command{git add #1}}

\newcommand{\gitadd}{\command{git add}}
\newcommand{\gitcommit}{\command{git commit}}
\newcommand{\gitpush}{\command{git push}}
\newcommand{\gitpull}{\command{git pull}}

\newcommand{\gitcommitmainprogram}{\command{git commit src/main/java/practicalnine/InvestigateRecursiveSum.java -m "Your
descriptive commit message"}}

% Use this when displaying a new command

\newcommand{\command}[1]{``\lstinline{#1}''}
\newcommand{\program}[1]{\lstinline{#1}}
\newcommand{\url}[1]{\lstinline{#1}}
\newcommand{\channel}[1]{\lstinline{#1}}
\newcommand{\option}[1]{``{#1}''}
\newcommand{\step}[1]{``{#1}''}

\usepackage{pifont}
\newcommand{\checkmark}{\ding{51}}
\newcommand{\naughtmark}{\ding{55}}

\usepackage{listings}
\lstset{
  basicstyle=\small\ttfamily,
  columns=flexible,
  breaklines=true
}

\usepackage{fancyvrb}
\usepackage{color}

\usepackage{fancyhdr}

\usepackage[margin=1in]{geometry}
\usepackage{fancyhdr}

\pagestyle{fancy}

\fancyhf{}
\rhead{Computer Science 111}
\lhead{Practical Assignment \assignmentnumber{}}
\rfoot{Page \thepage}
\lfoot{\duedate}

\usepackage{titlesec}
\titlespacing\section{0pt}{6pt plus 4pt minus 2pt}{4pt plus 2pt minus 2pt}

\newcommand{\labtitle}[1]
{
  \begin{center}
    \begin{center}
      \bf
      CMPSC 111\\Introduction to Computer Science I\\
      Fall 2017\\
      \medskip
    \end{center}
    \bf
    #1
  \end{center}
}

\begin{document}

\thispagestyle{empty}

\labtitle{Practical \assignmentnumber{} \\ \assigneddate{} \\ \duedate{}}

\section*{Objectives}

In this practical assignment, you will explore an already implemented Java
program that performs a recursive summation. As you modify the program, you will
run it and then observe, record, and comment on the output that is produced. The
goal for this assignment is to ensure that you understand how a Java program
recursively executes a method. Finally, you will continue to practice writing
technical documentation in Markdown and using a Git repository hosted by GitHub.

\section*{Suggestions for Success}

\begin{itemize}
  \setlength{\itemsep}{0pt}

\item {\bf Use the laboratory computers}. The computers in this laboratory feature specialized software for completing
  this course's laboratory and practical assignments. If it is necessary for you to work on a different machine, be sure
  to regularly transfer your work to a laboratory machine so that you can check its correctness. If you cannot use a
  laboratory computer and you need help with the configuration of your own laptop, then please carefully explain its
  setup to a teaching assistant or the course instructor when you are asking questions.

\item {\bf Follow each step carefully}. Slowly read each sentence in the assignment sheet, making sure that you
  precisely follow each instruction. Take notes about each step that you attempt, recording your questions and ideas and
  the challenges that you faced. If you are stuck, then please tell a teaching assistant or instructor what assignment
  step you recently completed.

\item {\bf Regularly ask and answer questions}. Please log into Slack at the start of a laboratory or practical session
  and then join the appropriate channel. If you have a question about one of the steps in an assignment, then you can
  post it to the designated channel. Or, you can ask a student sitting next to you or talk with a teaching assistant or
  the course instructor.

\item {\bf Store your files in GitHub}. As in all of your past assignments, you will be responsible for storing
  all of your files (e.g., Java source code and Markdown-based writing) in a Git repository using GitHub Classroom.
  Please verify that you have saved your source code in your Git repository by using \command{git status} to ensure that
  everything is updated. You can see if your assignment submission meets the established correctness requirements by
  using the provided checking tools on your local computer and in checking the commits in GitHub.

\item {\bf Keep all of your files}. Don't delete your programs, output files, and written reports after you submit them
  through GitHub; you will need them again when you study for the quizzes and examinations and work on the other
  laboratory, practical, and final project assignments.

\item {\bf Back up your files regularly}. All of your files are regularly backed-up to the servers in the Department of
  Computer Science and, if you commit your files regularly, stored on GitHub. However, you may want to use a flash
  drive, Google Drive, or your favorite backup method to keep an extra copy of your files on reserve. In the event of
  any type of system failure, you are responsible for ensuring that you have access to a recent backup copy of all your
  files.

\item {\bf Explore teamwork and technologies}. While certain aspects of these assignments will be challenging for you,
  each part is designed to give you the opportunity to learn both fundamental concepts in the field of computer science
  and explore advanced technologies that are commonly employed at a wide variety of companies. To explore and develop
  new ideas, you should regularly communicate with your team and/or the teaching assistants and tutors.

\item {\bf Hone your technical writing skills}. Computer science assignments require to you write technical
  documentation and descriptions of your experiences when completing each task. Take extra care to ensure that your
  writing is interesting and both grammatically and technically correct, remembering that computer scientists must
  effectively communicate and collaborate with their team members and the tutors, teaching assistants, and course
  instructor.

\item {\bf Review the Honor Code on the syllabus}. While you may discuss your assignments with others, copying source
  code or writing is a violation of Allegheny College's Honor Code.

\end{itemize}

\section*{Reading Assignment}

If you have not done so already, please read all of the relevant ``GitHub
Guides'', available at \url{https://guides.github.com/}, that explain how to use
many of GitHub's features. In particular, please make sure that you have read
guides such as ``Mastering Markdown'' and ``Documenting Your Projects on
GitHub''; each of them will help you to understand how to use both GitHub and
GitHub Classroom. To learn more about the concepts associated with exception
handling in the Java programming language, please study the content in Chapter
12. In particular, make sure that you understand the method that performs a
recursive summation in Section 12.2.

\section*{Using and Extending a Program that Recursively Sums}

To access the practical assignment, you should go into the \channel{\#announcements} channel in our Slack team and find
the announcement that provides a link for it. Copy this link and paste it into your web browser. Now, you should accept
the practical assignment and see that GitHub Classroom created a new GitHub repository for you to access the
assignment's starting materials and to store the completed version of your assignment. Specifically, to access your new
GitHub repository for this assignment, please click the green ``Accept'' button and then click the link that is prefaced
with the label ``Your assignment has been created here''. If you accepted the assignment and correctly followed these
steps, you should have created a GitHub repository with a name like
``Allegheny-Computer-Science-111-Fall-2017/computer-science-111-fall-2017-practical-10-gkapfham''. Unless you provide the
instructor with documentation of the extenuating circumstances that you are facing, not accepting the assignment means
that you automatically receive a failing grade for it. Please follow the steps from the previous laboratory assignments
for finding your ``home base'' for this practical assignment; see the instructor if you are stuck on getting started.

After reviewing the provided source code file, please use Gradle to build and run the \mainprogramsource{} program. At
the outset, you will see that this program does not throw any exceptions at all. So, what you should do next is to
incrementally and individually uncomment each of the calls to the {\tt throwsExceptions} method and observe and
understand the output from the program. The idea is that you will uncomment a method call in {\tt main}, build and run
the program, observe and understand the program's output, comment out that line again, and then move onto the next line
of code in the program's {\tt main} method.

Each time the program throws a different exception, make sure that you understand why it does so. Next, you will notice
that, in certain cases, the program does not output a full ``stack trace'' that prints out in the terminal window which
exception is thrown. As such, you may want to add code to the catch blocks of {\tt ExceptionExample} that can print the
stack trace. Finally, you must create a file called \reflection{} that explains the output from your various runs of the
program. Please write about what exception was thrown and explain why it was thrown.

\section*{Checking the Correctness of Your Program and Writing}

As in the past assignments, you are provided with an automated tool for checking the quality of your source code. Please
note that the practical assignments do not require you to produce a writing document as you do in the laboratory
assignments. However, to check your Java source code you can started with the use of GatorGrader, type the command
\gatorgraderstart{} in your terminal window. Once this step completes you can type \gatorgradercheck{}. If your work
does not meet all of the assignment's requirements, then you will see the following output in your terminal:
\command{Overall, are there any mistakes in the assignment? Yes}. If you do have mistakes in your assignment, then you
will need to review GatorGrader's output, find the mistake, and try to fix it. Specifically, don't forget to add in the
required writing! If you are having trouble running GatorGrader locally, don't forget to ensure that you still transfer
all of your source code to GitHub. Please see the course instructor if you have questions about this step.

Once your program is building correctly, fulfilling at least some of the assignment's requirements, you should transfer
your files to GitHub using the \gitcommit{} and \gitpush{} commands. For example, if you want to signal that the
\mainprogramsource{} file has been changed and is ready for transfer to GitHub you would first type
\gitcommitmainprogram{} in your terminal, followed by typing \gitpush{} and checking to see that the transfer to GitHub
is successful. If you notice that transferring your code to GitHub did not work correctly, then please try to determine
why, asking a teaching assistant or the course instructor for help, if necessary.

After the course instructor enables \step{continuous integration} with a system called Travis CI, when you use the
\gitpush{} command to transfer your source code to your GitHub repository, Travis CI will initialize a \step{build} of
your assignment, checking to see if it meets all of the requirements. Since this is another challenging practical
assignment and you are continuing to learn how to throw exceptions, don't become frustrated if you make a mistake.
Instead, use your mistakes as an opportunity for learning both about the necessary technology and the background and
expertise of the other students in the class, the teaching assistants, and the course instructor.

Students do not need to submit printed source code or technical writing for any assignment in this course.
Instead, this assignment invites you to submit, using GitHub, the following deliverables. Because this is a practical
assignment, you are not required to submit a separate ``reflection''.

\begin{enumerate}

\setlength{\itemsep}{0in}

\item A completed version of the \program{writing/exceptions.md} file that reports on the output and behavior of the
  provided program when it individually triggers each exception.

\end{enumerate}

\section*{Evaluation of Your Practical Assignment}

Practical assignments are graded on a completion --- or ``checkmark'' --- basis. If your GitHub repository has a
\checkmark{} for the last commit before the deadline then you will receive the highest possible grade for the
assignment. However, you will fail the assignment if you do not complete it correctly, as evidenced by either a red
\naughtmark{} in your commit listing or the absence of a functioning GitHub repository for this practical assignment,
after the set deadline for completing the project. Please see the course instructor if you do not understand how
practical assignments are graded or you do not know how to complete one of the specific tasks in this assignment.
Finally, remember that, in adherence to the Honor Code, students should complete this practical assignment on an
individual basis. Please see the course instructor if you have any questions about this course policy.

\end{document}
