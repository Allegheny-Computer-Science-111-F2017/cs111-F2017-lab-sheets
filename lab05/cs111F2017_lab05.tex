\documentclass[11pt]{article}

% NOTE: The "Edit" sections are changed for each assignment

% Edit these commands for each assignment

\newcommand{\assignmentduedate}{October 5}
\newcommand{\assignmentassignedate}{October 28}
\newcommand{\assignmentnumber}{Five}

\newcommand{\labyear}{2017}
\newcommand{\labday}{Thursday}
\newcommand{\labtime}{2:30 pm}

\newcommand{\assigneddate}{Assigned: \labday, \assignmentassignedate, \labyear{} at \labtime{}}
\newcommand{\duedate}{Due: \labday, \assignmentduedate, \labyear{} at \labtime{}}

% Edit these commands to give the name to the main program

\newcommand{\mainprogram}{\lstinline{WordHide}}
\newcommand{\mainprogramsource}{\lstinline{src/main/java/labfive/WordHide.java}}
\newcommand{\secondprogram}{\lstinline{PaintGraphicalScene}}
\newcommand{\secondprogramsource}{\lstinline{src/main/java/labfive/PaintGraphicalScene.java}}

% Edit this commands to describe key deliverables

\newcommand{\reflection}{\lstinline{writing/reflection.md}}

% Commands to describe key development tasks

% --> Running gatorgrader.sh
\newcommand{\gatorgraderstart}{\command{./gatorgrader.sh --start}}
\newcommand{\gatorgradercheck}{\command{./gatorgrader.sh --check}}

% --> Compiling and running program with gradle
\newcommand{\gradlebuild}{\command{gradle build}}
\newcommand{\gradlerun}{\command{gradle run}}

% Commands to describe key git tasks

\newcommand{\gitcommitfile}[1]{\command{git commit #1}}
\newcommand{\gitaddfile}[1]{\command{git add #1}}

\newcommand{\gitadd}{\command{git add}}
\newcommand{\gitcommit}{\command{git commit}}
\newcommand{\gitpush}{\command{git push}}
\newcommand{\gitpull}{\command{git pull}}

\newcommand{\gitaddmainprogram}{\command{git add src/main/java/labfive/WordHide.java}}
\newcommand{\gitcommitmainprogram}{\command{git commit src/main/java/labfive/WordHide.java -m "Your
descriptive commit message"}}

% Use this when displaying a new command

\newcommand{\command}[1]{``\lstinline{#1}''}
\newcommand{\program}[1]{\lstinline{#1}}
\newcommand{\url}[1]{\lstinline{#1}}
\newcommand{\channel}[1]{\lstinline{#1}}
\newcommand{\option}[1]{``{#1}''}
\newcommand{\step}[1]{``{#1}''}

\usepackage{pifont}
\newcommand{\checkmark}{\ding{51}}
\newcommand{\naughtmark}{\ding{55}}

\usepackage{listings}
\lstset{
  basicstyle=\small\ttfamily,
  columns=flexible,
  breaklines=true
}

\usepackage{fancyhdr}

\usepackage[margin=1in]{geometry}
\usepackage{fancyhdr}

\pagestyle{fancy}

\fancyhf{}
\rhead{Computer Science 111}
\lhead{Laboratory Assignment \assignmentnumber{}}
\rfoot{Page \thepage}
\lfoot{\duedate}

\usepackage{titlesec}
\titlespacing\section{0pt}{6pt plus 4pt minus 2pt}{4pt plus 2pt minus 2pt}

\newcommand{\labtitle}[1]
{
  \begin{center}
    \begin{center}
      \bf
      CMPSC 111\\Introduction to Computer Science I\\
      Fall 2017\\
      \medskip
    \end{center}
    \bf
    #1
  \end{center}
}

\begin{document}

\thispagestyle{empty}

\labtitle{Laboratory \assignmentnumber{} \\ \assigneddate{} \\ \duedate{}}

\section*{Objectives}

In addition to enhancing the skills that you have learned in the past laboratory assignments, the purpose of this
assignment is to explore the ideas of a ``class'' and an ``object'' in the Java programming language.  Also, you will
learn how to use the methods provided by the {\tt java.lang.String} class to inspect and manipulate a {\tt String}
object. You will then apply your knowledge of {\tt String}s to the application domain of steganography, or the practice
of ``hiding'' messages inside of non-secret content. Finally, you will continue to use development tools to complete a
real-world programming task, using sites like GitHub and Slack to support your completion of the laboratory assignment.

\section*{Suggestions for Success}

\begin{itemize}
  \setlength{\itemsep}{0pt}

\item {\bf Use the laboratory computers}. The computers in this laboratory feature specialized software for completing
  this course's laboratory and practical assignments. If it is necessary for you to work on a different machine, be sure
  to regularly transfer your work to a laboratory machine so that you can check its correctness. If you cannot use a
  laboratory computer and you need help with the configuration of your own laptop, then please carefully explain its
  setup to a teaching assistant or the course instructor when you are asking questions.

\item {\bf Follow each step carefully}. Slowly read each sentence in the assignment sheet, making sure that you
  precisely follow each instruction. Take notes about each step that you attempt, recording your questions and ideas and
  the challenges that you faced. If you are stuck, then please tell a teaching assistant or instructor what assignment
  step you recently completed.

\item {\bf Regularly ask and answer questions}. Please log into Slack at the start of a laboratory or practical session
  and then join the appropriate channel. If you have a question about one of the steps in an assignment, then you can
  post it to the designated channel. Or, you can ask a student sitting next to you or talk with a teaching assistant or
  the course instructor.

\item {\bf Store your files in GitHub}. Starting with this laboratory assignment, you will be responsible for storing
  all of your files (e.g., Java source code and Markdown-based writing) in a Git repository using GitHub Classroom.
  Please verify that you have saved your source code in your Git repository by using \command{git status} to ensure that
  everything is updated. You can see if your assignment submission meets the established correctness requirements by
  using the provided checking tools on your local computer and in checking the commits in GitHub.

\item {\bf Keep all of your files}. Don't delete your programs, output files, and written reports after you submit them
  through GitHub; you will need them again when you study for the quizzes and examinations and work on the other
  laboratory, practical, and final project assignments.

\item {\bf Back up your files regularly}. All of your files are regularly backed-up to the servers in the Department of
  Computer Science and, if you commit your files regularly, stored on GitHub. However, you may want to use a flash
  drive, Google Drive, or your favorite backup method to keep an extra copy of your files on reserve. In the event of
  any type of system failure, you are responsible for ensuring that you have access to a recent backup copy of all your
  files.

\item {\bf Explore teamwork and technologies}. While certain aspects of the laboratory assignments will be challenging
  for you, each part is designed to give you the opportunity to learn both fundamental concepts in the field of computer
  science and explore advanced technologies that are commonly employed at a wide variety of companies. To explore and
  develop new ideas, you should regularly communicate with your team and/or the teaching assistants and tutors.

\item {\bf Hone your technical writing skills}. Computer science assignments require to you write technical
  documentation and descriptions of your experiences when completing each task. Take extra care to ensure that your
  writing is interesting and both grammatically and technically correct, remembering that computer scientists must
  effectively communicate and collaborate with their team members and the tutors, teaching assistants, and course
  instructor.

\item {\bf Review the Honor Code on the syllabus}. While you may discuss your assignments with others, copying source
  code or writing is a violation of Allegheny College's Honor Code.

\end{itemize}

\section*{Reading Assignment}

To review what you have already learned about about variables, expressions, and user input, please read Sections
2.1--2.6 in your textbook; pay close attention to the {\tt Scanner} methods in Figure 2.7 and the program in Listing
2.8. To learn more about Java classes and objects and, in particular, the methods provided by the {\tt java.lang.String}
class, please study Sections 3.1--3.2 in the textbook. You should also study all of the slides that we discussed during
class sessions. Finally, don't forget to examine the ``Git Guides'' if you have questions about how to create and
use a Git repository. Please see the course instructor if you have any questions about these readings.

\section*{Accessing the Laboratory Assignment on GitHub}

To access the laboratory assignment, you should go into the \channel{\#announcements} channel in our Slack team and find
the announcement that provides a link for it. Copy this link and paste it into your web browser. Now, you should accept
the laboratory assignment and see that GitHub Classroom created a new GitHub repository for you to access the
assignment's starting materials and to store the completed version of your assignment. Specifically, to access your new
GitHub repository for this assignment, please click the green ``Accept'' button and then click the link that is prefaced
with the label ``Your assignment has been created here''. If you accepted the assignment and correctly followed these
steps, you should have created a GitHub repository with a name like
``Allegheny-Computer-Science-111-Fall-2017/computer-science-111-fall-2017-lab-5-gkapfham''. Unless you provide the
instructor with documentation of the extenuating circumstances that you are facing, not accepting the assignment means
that you automatically receive a failing grade for it.

Before you move to the next step of this assignment, please make sure that you read all of the content on the web site
for your new GitHub repository, paying close attention to the technical details about the commands that you will type
and the output that your program must produce. Now you are ready to download the starting materials to your laboratory
computer. Click the ``Clone or download'' button and, after ensuring that you have selected ``Clone with SSH'', please
copy this command to your clipboard. To enter into the right directory you should now type \command{cd cs111F2017}.
Next, you can type the command \command{ls} and see that there are some files or directories inside of this directory.
By typing \command{git clone} in your terminal and then pasting in the string that you copied from the GitHub site you
will download all of the code for this assignment. For instance, if the course instructor ran the \command{git clone}
command in the terminal, it would look like:

\begin{lstlisting}
  git clone https://github.com/Allegheny-Computer-Science-111-F2017/computer-science-111-fall-2017-lab-5-gkapfham.git
\end{lstlisting}

After this command finishes, you can use \command{cd} to change into the new directory. If you want to \step{go back}
one directory from your current location, then you can type the command \command{cd ..}. Please continue to use the
\command{cd} and \command{ls} commands to explore the files that you automatically downloaded from GitHub. What files
and directories do you see? What do you think is their purpose? Spend some time exploring, sharing your discoveries with
a neighbor and a \mbox{teaching assistant}.

\subsection*{Implementing and Evaluating a Steganography Program}

Within the field of computer security, there are many sub-fields that develop strategies for sending, receiving, and
storing secret messages. Involving the hiding of a secret ``in plain sight'', steganography is one way in which you can
create and send a secret message. For this assignment, you and your partner will implement a program called {\tt
WordHide.java} that will perform these operations:

\vspace*{-.1in}

\begin{enumerate}

  \itemsep0in

  \item Prompt the user for a word that is exactly ten characters in length. For now, {\tt WordHide} should allow the
    user to enter any type of ``padding character'' before or after a word that is less than ten characters. If the word
    is longer than ten characters, then your program should simply discard all of the characters after the tenth one
    (you may also implement better strategies).

  \item Since the {\tt WordHide} program must output all of its character in a capitalized form, you should transform the
    word provided by the user so that it only contains upper-case letters.

  \item Finally, {\tt WordHide} should output a $20 \times 20$ ``grid'' of letters that contain the user's word
    ``hidden'' inside of it. All of the letters in this grid should be capitalized. You and your partner should
    brainstorm and prototype different techniques for effectively hiding the user's word in the grid of letters. As you
    implement your program, you must make decisions about the following matters: (i) what letters will you add to the
    grid to best hide the user's word? (ii) where will you place the user's word in the grid? (iii) what features of the
    Java programming language will you use to ensure the grid is formatted properly in the terminal? As you answer these
    questions and finish the implementation and testing of {\tt WordHide.java}, it may help to consider the fact that
    the user's word might be better hidden if the grid contains both some randomly chosen letters and some letters found
    in the user's input. Your team should devote a considerable amount of time to brainstorming, prototyping, and
    evaluating (e.g., by asking a friend to try your program) different ways in which you can effectively hide words.

\end{enumerate}

\vspace*{-.1in}

In summary, this assignment asks you to write a program, \mainprogramsource{}, that will complete a word-hiding task. In
particular, it must meet the following requirements:

\vspace*{-.1in}
\begin{enumerate}
  \setlength{\itemsep}{0pt}

\item Contain at least four single-line comments and two multi-line comments.
\item Declare and use the \command{CANVAS_HEIGHT} variable.
\item Declare and use the \command{CANVAS_WIDTH} variable.

\end{enumerate}

Whenever you are finished typing text Java code in \program{gvim}, press the ESC key located in the upper left corner of
the keyboard. This should remove the word \command{--INSERT--} from the bottom of the screen and take you out of insert
mode. Use the \option{File/Save} menu to save your program. Alternatively, if you would like to use the keyboard to save
your file, you can press \command{:w} when you are not in insert mode. Leaving the {\tt gvim} window open, go back to
your terminal window.

If you want to \step{build} your program you can type the command \gradlebuild{} in your terminal, thereby causing the
Java compiler to check your program for errors and get it ready to run. If you get any error messages, go back into
\program{gvim} and try to figure out what you mis-typed and fix it. Once you have solved the problem, make a note of the
error and the solution for resolving it. Re-save your program and then build it again by re-running the \gradlebuild{}.
If you cannot build \mainprogram{} correctly, then please talk with a teaching assistant or the instructor.

When all of the errors are eliminated, you can run your program by typing \gradlerun{} in the terminal window---this is
the ``execute'' step that will run your program and produce the designated output. You should see your name, today's
date, and the graphical output. Make sure there are spaces separating words in your output (did you forget to put a
space inside the quotation marks after your last name?). If not, then repair the program and re-build and re-run it.
Once the program runs, please reflect on this process. What step did you find to be the most challenging? Why? You
should write your reflections in a file, called \reflection{}, that uses the Markdown writing language. To complete this
aspect of the assignment, you should write one high-quality paragraph that reports on your experiences with the various
commands and Java code segments.

\section*{Checking the Correctness of Your Program and Writing}

As verified by Checkstyle, the code for the \mainprogramsource{} and the \secondprogramsource{} files must adhere to all
of the requirements in the Google Java Style Guide available at
\url{https://google.github.io/styleguide/javaguide.html}. The Markdown file that contains your reflection must adhere to
the standards described in the Markdown Syntax Guide \url{https://guides.github.com/features/mastering-markdown/}.
Finally, your \reflection{} file should adhere to the Markdown standards established by the \step{Markdown linting} tool
available at \url{https://github.com/markdownlint/markdownlint/} and the writing standards set by the \step{prose
linting} tool from \url{http://proselint.com/}. Instead of requiring you to manually check that your deliverables adhere
to these industry-accepted standards, the GatorGrader tool that you will use in this laboratory assignment makes it easy
for you to automatically check if your submission meets these well-established standards for correctness.

Since this is not your first laboratory assignment, you will notice that the provide source code does not contain all of
the required comments at the top of the Java source code file. This means that you will have to inspect the source code
from previous laboratory and practical assignments to review how to create the comments in the \mainprogramsource{} and
\secondprogramsource{} files. Moreover, the provided source code is missing many of the lines that are needed to pass
the GatorGrader checks. Please review the requirements for these two Java source code files, as outlined in the previous
section. You should also study the source code of these files to learn more about what you need to add to them.

To get started with the use of GatorGrader, type the command \gatorgraderstart{} in your terminal window. Once this step
completes you can type \gatorgradercheck{}. If your work does not meet all of the assignment's requirements, then you
will see the following output in your terminal: \command{Overall, are there any mistakes in the assignment? Yes}. If you
do have mistakes in your assignment, then you will need to review GatorGrader's output, find the mistake, and try to fix
it. Once your program is building correctly, fulfilling at least some of the assignment's requirements, you should
transfer your files to GitHub using the \gitcommit{} and \gitpush{} commands. For example, if you want to signal that
the \mainprogramsource{} file has been changed and is ready for transfer to GitHub you would first type
\gitcommitmainprogram{} in your terminal, followed by typing \gitpush{}, and then checking to see that the transfer to
GitHub is successful. Remember, to correctly complete this assignment you need to commit all code and writing files to
GitHub. If you notice that the network communication with GitHub did not work, then please try to determine why, asking
a teaching assistant or the instructor for assistance.

After the course instructor enables \step{continuous integration} with a system called Travis CI, when you use the
\gitpush{} command to transfer your source code to your GitHub repository, Travis CI will initialize a \step{build} of
your assignment, checking to see if it meets all of the requirements. If both your source code and writing meet all of
the established requirements, then you will see a green \checkmark{} in the listing of commits in GitHub after awhile.
If your submission does not meet the requirements, a red \naughtmark{} will appear instead. The instructor will reduce a
student's grade for this assignment if the red \naughtmark{} appears on the last commit in GitHub immediately before the
assignment's due date. Yet, if the green \checkmark{} appears on the last commit in your GitHub repository, then you
satisfied all of the main checks, thereby allowing the course instructor to evaluate other aspects of your source code
and writing, as further described in the \step{Evaluation} section of this assignment sheet. Unless you provide the
instructor with documentation of the extenuating circumstances that you are facing, no late work will be considered
towards your grade for this laboratory assignment. In conclusion, here are some points to remember for creating programs
that display graphical output:

\begin{enumerate}
  \setlength{\itemsep}{0pt}

\item You should think carefully about how the $20 \times 20$ grid can be displayed using variables.

\item As in past assignments, your program only needs to have one {\tt main} method in one file.

\item See Figure 3.1 for a listing of some common \command{String} methods for use in your program.

\item Your program will alternate between creating and displaying textual output---this is okay!

\item Don't forget to review the assignment sheets from the previous laboratory and practical assignments as they
  contain insights that will support your completion of this assignment.

\end{enumerate}


\section*{Summary of the Required Deliverables}

\noindent Students do not need to submit printed source code or technical writing for any assignment in this course.
Instead, this assignment invites you to submit, using GitHub, the following deliverables.

\begin{enumerate}

  \setlength{\itemsep}{0in}

\item Stored in \reflection{}, a two-paragraph reflection on the commands that you typed in \command{gvim} and the
  terminal window. This Markdown-based document should explain the input, output, and behavior of each command and the
  challenges that you confronted when using it. For every challenge that you encountered, please explain your solution
  for it. This file should also explain your strategy for hiding a word in the $20 \times 20$ character grid.

\item A complete and correct version of \mainprogramsource{} that both meets all of the established requirements and
  produces the desired output.

\end{enumerate}

\section*{Evaluation of Your Laboratory Assignment}

Using a report that the instructor shares with you through the commit log in GitHub, you will privately received a grade
on this assignment and feedback on your submitted deliverables. Your grade for the assignment will be a function of the
whether or not it was submitted in a timely fashion and if your program received a green \checkmark{} indicating that it
met all of the requirements. Other factors will also influence your final grade on the assignment. In addition to
studying the efficiency and effectiveness of your Java source code, the instructor will also evaluate the accuracy of
both your technical writing and the comments in your source code. If your submission receives a red \naughtmark{}, the
instructor will reduce your grade for the assignment while still considering the regularity with which you committed to
your GitHub repository and the overall quality of your partially completed work. Please see the instructor if you have
questions about the evaluation of this laboratory assignment.

% \section*{Adhering to the Honor Code}

% In adherence to the Honor Code, students should complete this assignment on an individual basis. While it is appropriate
% for students in this class to have high-level conversations about the assignment, it is necessary to distinguish
% carefully between the student who discusses the principles underlying a problem with others and the student who produces
% assignments that are identical to, or merely variations on, someone else's work. Deliverables (e.g., Java source code or
% Markdown-based technical writing) that are nearly identical to the work of others will be taken as evidence of violating
% the \mbox{Honor Code}. Please see the course instructor if you have questions about this policy.

\end{document}
