\documentclass[11pt]{article}

% NOTE: The "Edit" sections are changed for each assignment

% Edit these commands for each assignment

\newcommand{\assignmentduedate}{October 26}
\newcommand{\assignmentassignedate}{October 19}
\newcommand{\assignmentnumber}{Seven}

\newcommand{\labyear}{2017}
\newcommand{\labday}{Thursday}
\newcommand{\labtime}{2:30 pm}

\newcommand{\assigneddate}{Assigned: \labday, \assignmentassignedate, \labyear{} at \labtime{}}
\newcommand{\duedate}{Due: \labday, \assignmentduedate, \labyear{} at \labtime{}}

% Edit these commands to give the name to the main program

\newcommand{\mainprogram}{\lstinline{CommandLineGeometer}}
\newcommand{\mainprogramsource}{\lstinline{src/main/java/labseven/CommandLineGeometer.java}}

\newcommand{\secondprogram}{\lstinline{GeometricCalculator}}
\newcommand{\secondprogramsource}{\lstinline{src/main/java/labseven/GeometricCalculator.java}}

% Edit this commands to describe key deliverables

\newcommand{\reflection}{\lstinline{writing/reflection.md}}

% Commands to describe key development tasks

% --> Running gatorgrader.sh
\newcommand{\gatorgraderstart}{\command{./gatorgrader.sh --start}}
\newcommand{\gatorgradercheck}{\command{./gatorgrader.sh --check}}

% --> Compiling and running program with gradle
\newcommand{\gradlebuild}{\command{gradle build}}
\newcommand{\gradlerun}{\command{gradle run}}

% Commands to describe key git tasks

\newcommand{\gitcommitfile}[1]{\command{git commit #1}}
\newcommand{\gitaddfile}[1]{\command{git add #1}}

\newcommand{\gitadd}{\command{git add}}
\newcommand{\gitcommit}{\command{git commit}}
\newcommand{\gitpush}{\command{git push}}
\newcommand{\gitpull}{\command{git pull}}

\newcommand{\gitaddmainprogram}{\command{git add src/main/java/labseven/CommandLineGeometer.java}}
\newcommand{\gitcommitmainprogram}{\command{git commit src/main/java/labseven/CommandLineGeometer.java -m "Your
descriptive commit message"}}

% Use this when displaying a new command

\newcommand{\command}[1]{``\lstinline{#1}''}
\newcommand{\program}[1]{\lstinline{#1}}
\newcommand{\url}[1]{\lstinline{#1}}
\newcommand{\channel}[1]{\lstinline{#1}}
\newcommand{\option}[1]{``{#1}''}
\newcommand{\step}[1]{``{#1}''}

\usepackage{pifont}
\newcommand{\checkmark}{\ding{51}}
\newcommand{\naughtmark}{\ding{55}}

\usepackage{listings}
\lstset{
  basicstyle=\small\ttfamily,
  columns=flexible,
  breaklines=true
}

\usepackage{fancyhdr}

\usepackage[margin=1in]{geometry}
\usepackage{fancyhdr}

\pagestyle{fancy}

\fancyhf{}
\rhead{Computer Science 111}
\lhead{Laboratory Assignment \assignmentnumber{}}
\rfoot{Page \thepage}
\lfoot{\duedate}

\usepackage{titlesec}
\titlespacing\section{0pt}{6pt plus 4pt minus 2pt}{4pt plus 2pt minus 2pt}

\newcommand{\labtitle}[1]
{
  \begin{center}
    \begin{center}
      \bf
      CMPSC 111\\Introduction to Computer Science I\\
      Fall 2017\\
      \medskip
    \end{center}
    \bf
    #1
  \end{center}
}

\begin{document}

\thispagestyle{empty}

\labtitle{Laboratory \assignmentnumber{} \\ \assigneddate{} \\ \duedate{}}

\section*{Objectives}

In this laboratory assignment, you will learn more about using the {\tt java.lang.Math} class to perform numerical
calculations, further explore the creation of formatted output, learn how to use enumerated types, and practice calling
methods in another Java class. Additionally, since real-world software developers often have to debug source code
created by other developers and add features to existing code, you will participate in a ``bug hunt'' and add new source
code to an existing system. You will also learn how to use GitHub to effectively collaborate with team members.
Ultimately, you will create a working program, comprised of two Java classes, that performs calculations.

\section*{Suggestions for Success}

\begin{itemize}
  \setlength{\itemsep}{0pt}

\item {\bf Use the laboratory computers}. The computers in this laboratory feature specialized software for completing
  this course's laboratory and practical assignments. If it is necessary for you to work on a different machine, be sure
  to regularly transfer your work to a laboratory machine so that you can check its correctness. If you cannot use a
  laboratory computer and you need help with the configuration of your own laptop, then please carefully explain its
  setup to a teaching assistant or the course instructor when you are asking questions.

\item {\bf Follow each step carefully}. Slowly read each sentence in the assignment sheet, making sure that you
  precisely follow each instruction. Take notes about each step that you attempt, recording your questions and ideas and
  the challenges that you faced. If you are stuck, then please tell a teaching assistant or instructor what assignment
  step you recently completed.

\item {\bf Regularly ask and answer questions}. Please log into Slack at the start of a laboratory or practical session
  and then join the appropriate channel. If you have a question about one of the steps in an assignment, then you can
  post it to the designated channel. Or, you can ask a student sitting next to you or talk with a teaching assistant or
  the course instructor.

\item {\bf Store your files in GitHub}. As in the past laboratory assignments, you will be responsible for storing all
  of your files (e.g., Java source code and Markdown-based writing) in a Git repository using GitHub Classroom. Please
  verify that you have saved your source code in your Git repository by using \command{git status} to ensure that
  everything is updated. You can see if your assignment submission meets the established correctness requirements by
  using the provided checking tools on your local computer and by checking the commits in GitHub.

\item {\bf Keep all of your files}. Don't delete your programs, output files, and written reports after you submit them
  through GitHub; you will need them again when you study for the quizzes and examinations and work on the other
  laboratory, practical, and final project assignments.

\item {\bf Back up your files regularly}. All of your files are regularly backed-up to the servers in the Department of
  Computer Science and, if you commit your files regularly, stored on GitHub. However, you may want to use a flash
  drive, Google Drive, or your favorite backup method to keep an extra copy of your files on reserve. In the event of
  any type of system failure, you are responsible for ensuring that you have access to a recent backup copy of all your
  files.

\item {\bf Explore teamwork and technologies}. While certain aspects of the laboratory assignments will be challenging
  for you, each part is designed to give you the opportunity to learn both fundamental concepts in the field of computer
  science and explore advanced technologies that are commonly employed at a wide variety of companies. To explore and
  develop new ideas, you should regularly communicate with your team and/or the teaching assistants and tutors.

\item {\bf Hone your technical writing skills}. Computer science assignments require to you write technical
  documentation and descriptions of your experiences when completing each task. Take extra care to ensure that your
  writing is interesting and both grammatically and technically correct, remembering that computer scientists must
  effectively communicate and collaborate with their team members and the tutors, teaching assistants, and course
  instructor.

\item {\bf Review the Honor Code on the syllabus}. While you may discuss your assignments with others, copying source
  code or writing is a violation of Allegheny College's Honor Code.

\end{itemize}

\section*{Reading Assignment}

After reviewing the GitHub materials, all of the assignment sheets for the past laboratory and practical sessions, and
the course slides and notes, you should review Sections 3.5 through 3.8 of your textbook. To enhance your understanding
of some points in this lab you should additionally examine Figures 4.7 and 4.8. See the instructor if you have any
questions about these readings.

\section*{Accessing the Laboratory Assignment on GitHub}

To access the laboratory assignment, you should go into the \channel{\#announcements} channel in our Slack team and find
the announcement that provides a link for it. Now, make sure that the leader of your team also notes your team number
and first copies this link and pastes it into their web browser. Next, the team leader (i.e., the first person in the
group listing on Slack) will create their team with the name \command{Computer-Science-111-Fall-2017-Lab-7-Group-<group
number>} and then accept the laboratory assignment and see that GitHub Classroom created a new GitHub repository for
your team to access the assignment's starting materials and to store the completed version of your assignment. Note that
the team leader will have to type their group name and number into a text field. For instance, if the team leader was in
the second group then that person would type ``Group 2'' into the text field. At this point, each additional member of
the team can accept the assignment through GitHub. Please make sure that each of your team members joins the team to
which the instructor assigned them to work. Unless you provide the instructor with documentation of the extenuating
circumstances that you are facing, not accepting the assignment means that you automatically receive a failing grade for
it. Please see the course instructor with any questions.

Before you move to the next step of this assignment, please make sure that you read all of the content on the web site
for your new GitHub repository, paying close attention to the technical details about the commands that you will type
and the output that your program must produce. Now you are ready to download the starting materials to your laboratory
computer. Click the ``Clone or download'' button and, after ensuring that you have selected ``Clone with SSH'', please
copy this command to your clipboard. To enter into the right directory you should now type \command{cd cs111F2017}.
Next, you can type the command \command{ls} and see that there are some files or directories inside of this directory.
By typing \command{git clone} in your terminal and then pasting in the string that you copied from the GitHub site you
will download all of the code for this assignment. For instance, if the course instructor ran the \command{git clone}
command in the terminal, it would look like:

\begin{lstlisting}
  git clone git@github.com:Allegheny-Computer-Science-111-F2017/computer-science-111-fall-2017-lab-7-group-2.git
\end{lstlisting}

After this command finishes, you can use \command{cd} to change into the new directory. If you want to \step{go back}
one directory from your current location, then you can type the command \command{cd ..}. Please continue to use the
\command{cd} and \command{ls} commands to explore the files that you automatically downloaded from GitHub. What files
and directories do you see? What do you think is their purpose? Spend some time exploring, sharing your discoveries with
a neighbor and a \mbox{teaching assistant}.

\section*{Participating in a Collaborative ``Bug Hunt''}

You and your group members should explore your repository by using {\tt gvim} to study the source code of the {\tt
CommandLineGeometer.java} and {\tt GeometricCalculator.java} files. What methods do these classes provide? How do they
work? Does any of this code look incorrect? Why?

After carefully reviewing the source code and PP 3.6, 3.7, and 3.9 on page 158 of your textbook and then compiling and
running the {\tt CommandLineGeometer} class, you should notice that there are several defects in this program. As such,
you will need to take part in a ``bug hunt'' to find and fix all of the problems. First, you should find the method
responsible for calculating the volume of a sphere. Using the equation in PP 3.6 as a reference point, what is the
defect in this method?

The {\tt GeometricCalculator} also provides a method to calculate a triangle's area. Once again, there is a mistake in
this method. Can you find and fix it? How did you know that this was the bug? If you investigate the source code of the
method for calculating the volume of a cylinder, you will notice that there is another defect lurking in the source
code. Wait! If you carefully study the way in which the {\tt CommandLineGeometer} calls the method provided by the {\tt
GeometricCalculator} and then displays the resulting output, you will realize that there is another bug in this program.
Make sure you have found all of the problems before continuing with this laboratory assignment.

You and your group member(s) should complete this assignment by using GitHub and proceeding incrementally. That is, one
member of the team should be in charge of typing when you find and fix the first defect. Then, that person should commit
and push all of the repaired Java files and the other team member(s) should run \command{git pull}. Now, a new team
member should perform the typing when finding and fixing the next defect. Importantly, one of the goals for this
assignment is for all of the team members to gain experience with using GitHub to collaborate in a team. Please see the
course instructor if you do not understand how to use GitHub to complete this task.

\section*{Extending the Geometry Calculator}

\begin{sloppypar}
  After reviewing the aforementioned programming projects on page 158 of your textbook, you will also notice that the {\tt
  GeometricCalculator.java} does not contain methods for calculating the surface area of a sphere or a cylinder. While
  avoiding all of the mistake types that you corrected in the previous phase of this assignment, please add in new methods
  to perform these calculations. In addition, you will need to add appropriate input and output statements and method
  calls to the {\tt CommandLineGeometer} to ensure that the entire program works correctly. For instance, you will need to
  implement a new method called {\tt calculateSphereSurfaceArea} to the {\tt GeometricCalculator.java} file and then add
  input and output code to the {\tt CommandLineGeometer.java} file. Whenever possible, try to follow the correct
  pattern established in the given source code. Please see the course instructor if you have questions!
\end{sloppypar}

As you continue to critically review the source code of the {\tt CommandLineGeometer}, you will notice that it does not
always consistently produce output for the user. For example, even though it displays the user-input radius before
calling {\tt calculateSphereVolume}, it does not appropriately display the sides of the triangle for the {\tt
computeTriangleArea} method---can you please add in this feature? Moreover, none of the output of the {\tt double}
variables in the {\tt CommandLineGeometer} is formatted in a consistent fashion. To solve this problem, you should use
one of the techniques described in Section 3.6 of your textbook to format the output of all decimals to contain four
decimal points. For instance, you may consider creating an instance of the {\tt DecimalFormat} class.

Please notice that most of the provided code is not commented. As part of this assignment, your team should add
detailed comments to the code, following the JavaDoc standard. Please see the instructor if you are not sure how or
where to add comments to your program. Don't forget that you and your partner should evenly divide up the work needed to
complete this assignment. This means that both of your user names should appear in the commit log of your GitHub
repository.

\section*{Exploring the Features of the Java Language}

The {\tt CommandLineGeometer} program uses an enumerated type, as described in Section 3.7, to store specific values in
a variable. Intuitively, an enumerated type allows a variable to take on one of a pre-specified set of values or levels.
In this case, the {\tt GeometricShape} enumerated type can take on three possible values. What are they? Why is it
useful to use this type of variable?

You will notice that this laboratory assignment organizes the methods into two separate classes, as you have seen in
past assignments and in-class exercises. In particular, the {\tt CommandLineGeometer} provides the user interface for
our program and the {\tt GeometricCalculator} furnishes the methods that perform the required computations.  If you want
to make changes to the way in which the program accepts input or produces output, then you will need to modify the {\tt
CommandLineGeometer}. Otherwise, if you want to modify the way in which the program performs a computation, or add a new
computation, then you must make changes to the {\tt GeometricCalculator}. Overall, these two Java classes complete their
work by following a pattern similar to that which is outlined in Figures 4.7 and 4.8 of your textbook. Please see the
instructor if you have questions about this approach.

Additionally, it is important to note that this assignment asks you to add new methods to the {\tt
GeometricCalculator.java} file. To complete this task, you should directly copy the pattern that you see in the
provided methods, only making changes to implement the new functionality. Also, these methods accept parameters, of
type {\tt double}, that are passed from the {\tt main} method in the {\tt CommandLineGeometer} to one of the
``calculate'' methods in the {\tt GeometricCalculator}. You should also notice that all of the methods return a {\tt
double} variable to the method that calls it. Intuitively, the parameters are the ``input'' to a method and the return
values are the ``output'' of the method. When you create the required new methods, you should follow the pattern of the
previously implemented method, ensuring that you have the same types of input and output.

\section*{Checking the Correctness of Your Program and Writing}

As verified by Checkstyle, the code for the \mainprogramsource{} file must adhere to all of the requirements in the
Google Java Style Guide available at \url{https://google.github.io/styleguide/javaguide.html}. The Markdown file that
contains your reflection must adhere to the standards described in the Markdown Syntax Guide
\url{https://guides.github.com/features/mastering-markdown/}. Finally, your \reflection{} file should adhere to the
Markdown standards established by the \step{Markdown linting} tool available at
\url{https://github.com/markdownlint/markdownlint/} and the writing standards set by the \step{prose linting} tool from
\url{http://proselint.com/}. Instead of requiring you to manually check that your deliverables adhere to these
industry-accepted standards, the GatorGrader tool that you will use in this laboratory assignment makes it easy for you
to automatically check if your submission meets these well-established standards for correctness. Please see the
instructor if you have questions about GatorGrader.

Since this is not your first laboratory assignment, you will notice that the provide source code does not contain all of
the required comments at the top of the Java source code file. This means that you will have to inspect the source code
from previous laboratory and practical assignments to review how to create the comments in the \mainprogramsource{}
file. Moreover, the provided source code is missing many of the lines that are needed to pass the GatorGrader checks.
Review the requirements for these Java code files, as outlined in the previous section. You can study the source code of
this file to learn more about what you need to add to it. Don't forget to look in your team's GitHub repository to learn
about GatorGrader's checks!

To get started with the use of GatorGrader, type the command \gatorgraderstart{} in your terminal window. Once this step
completes you can type \gatorgradercheck{}. If your work does not meet all of the assignment's requirements, then you
will see the following output in your terminal: \command{Overall, are there any mistakes in the assignment? Yes}. If you
do have mistakes in your assignment, then you will need to review GatorGrader's output, find the mistake, and try to fix
it. Once your program is building correctly, fulfilling at least some of the assignment's requirements, you should
transfer your files to GitHub using the \gitcommit{} and \gitpush{} commands. For example, if you want to signal that
the \mainprogramsource{} file has been changed and is ready for transfer to GitHub you would first type
\gitcommitmainprogram{} in your terminal, followed by typing \gitpush{}, and then checking to see that the transfer to
GitHub is successful. Remember, to correctly complete this assignment you need to commit all code and writing files to
GitHub. Also, all of the team members should practice using GitHub. If you notice that the network communication with
GitHub did not work, then please try to determine why, asking a teaching assistant or the course instructor for
additional assistance.

After the course instructor enables \step{continuous integration} with a system called Travis CI, when you use the
\gitpush{} command to transfer your source code to your GitHub repository, Travis CI will initialize a \step{build} of
your assignment, checking to see if it meets all of the requirements. If both your source code and writing meet all of
the established requirements, then you will see a green \checkmark{} in the listing of commits in GitHub after awhile.
If your submission does not meet the requirements, a red \naughtmark{} will appear instead. The instructor will reduce a
student's grade for this assignment if the red \naughtmark{} appears on the last commit in GitHub immediately before the
assignment's due date. Yet, if the green \checkmark{} appears on the last commit in your GitHub repository, then you
satisfied all of the main checks, thereby allowing the course instructor to evaluate other aspects of your source code
and writing, as further described in the \step{Evaluation} section of this assignment sheet. Unless you provide the
instructor with documentation of the extenuating circumstances that you are facing, no late work will be considered
towards your grade for this laboratory assignment. In conclusion, here are some points to remember for creating programs
that performs computations:

\begin{enumerate}
  \setlength{\itemsep}{0pt}

\item The provided source code contains many defects in it---make sure that you find them all!

\item The provide source code is incomplete---make sure that you add all of the needed features.

\item Your team should draw a technical diagram to show how relationships between Java classes.

\item You should think carefully about how to correctly use formal and actual parameters.

\item As in past assignments, your program only needs to have one {\tt main} method in one file.

\item Since your program will need to use different data types, please learn more about them.

\item Your program may need to read in additional values from the provided text file.

\item Your program will alternate between creating and displaying textual output---this is okay!

\item Don't forget that you and your team member(s) should all contribute evenly by typing in text and using commands,
  like \command{git push} and \command{git pull}, to transfer files to and from GitHub.

\item Don't forget to review the assignment sheets from the previous laboratory and practical assignments as they
  contain insights that will support your completion of this assignment.

\end{enumerate}

\section*{Summary of the Required Deliverables}

\noindent Students do not need to submit printed source code or technical writing for any assignment in this course.
Instead, this assignment invites you to submit, using GitHub, the following deliverables.

\begin{enumerate}

  \setlength{\itemsep}{0in}

\item Stored in \reflection{}, a three-paragraph reflection on the commands that you typed in \command{gvim} and the
  terminal window. This Markdown-based document should explain the input, output, and behavior of each command and the
  challenges that you confronted when using it. For every challenge that you encountered, please explain your solution
  for it. This document should also explain how your team collaborated to finish the assignment, with each team member
  writing their own paragraph inside of this Markdown file.

\item A complete and correct version of \mainprogramsource{} that meets all of the set requirements and produces the
  desired textual output in the terminal.

\item A complete and correct version of \secondprogramsource{} that meets all of the set requirements and supports the
  desired textual output in the terminal.

\item The commit log in your GitHub repository should clearly show that the team members effectively collaborated. That
  is, the commit log should show that commits were evenly made by all team members; the photograph of each member should
  appear in the commit log.

\end{enumerate}

\section*{Evaluation of Your Laboratory Assignment}

Using a report that the instructor shares with you through the commit log in GitHub, you will privately received a grade
on this assignment and feedback on your submitted deliverables. Your grade for the assignment will be a function of the
whether or not it was submitted in a timely fashion and if your program received a green \checkmark{} indicating that it
met all of the requirements. Other factors will also influence your final grade on the assignment. In addition to
studying the efficiency and effectiveness of your Java source code, the instructor will also evaluate the accuracy of
both your technical writing and the comments in your source code. If your submission receives a red \naughtmark{}, the
instructor will reduce your grade for the assignment while still considering the regularity with which you committed to
your GitHub repository and the overall quality of your partially completed work. Please see the instructor if you have
questions about the evaluation of this laboratory assignment.

All team members will receive the same baseline grade for the laboratory assignment. If there are extenuating
circumstances in which one or more of the team members do not effectively collaborate to complete this assignment, then
the course instructor will adjust the grade of specific team members so that it is higher or lower than the baseline
grade, as is fair and necessary. Please see the instructor if you do not understand how he assigns grades for
collaborative assignments.

\section*{Adhering to the Honor Code}

In adherence to the Honor Code, students should complete this assignment with their team members. While it is
appropriate for students in this class to have high-level conversations about the assignment, it is necessary to
distinguish carefully between the student who discusses the principles underlying a problem with others and the student
who produces assignments that are identical to, or merely variations on, someone else's work. Deliverables (e.g., source
code or Markdown-based technical writing) that are nearly identical to the work of outsiders will be taken as evidence
of violating the \mbox{Honor Code}. Please see the course instructor with any questions about this policy.

\end{document}
