\documentclass[11pt]{article}

\newcommand{\assignmentduedate}{September 7}
\newcommand{\assignmentassignedate}{August 31}
\newcommand{\assignmentnumber}{One}

\newcommand{\labyear}{2017}
\newcommand{\labday}{Thursday}
\newcommand{\labtime}{2:30 pm}

\newcommand{\assigneddate}{Assigned: \labday, \assignmentassignedate, \labyear{} at \labtime{}}
\newcommand{\duedate}{Due: \labday, \assignmentduedate, \labyear{} at \labtime{}}

\usepackage{fancyhdr}
\usepackage{url}

\usepackage[margin=1in]{geometry}
\usepackage{fancyhdr}

\pagestyle{fancy}

\fancyhf{}
\rhead{Computer Science 111}
\lhead{Laboratory Assignment \assignmentnumber{}}
\rfoot{Page \thepage}
\lfoot{\duedate}

\usepackage{titlesec}
\titlespacing\section{0pt}{6pt plus 4pt minus 2pt}{4pt plus 2pt minus 2pt}

\newcommand{\labtitle}[1]
{
  \begin{center}
    \begin{center}
      \bf
      CMPSC 111\\Introduction to Computer Science I\\
      Fall 2017\\
      \medskip
    \end{center}
    \bf
    #1
  \end{center}
}

\begin{document}

\thispagestyle{empty}

\labtitle{Laboratory 1 \\ \assigneddate{} \\ \duedate{}}

\section*{Objective}

To learn how to use GitHub to access the files for a laboratory assignment. Additionally, to learn how to use the Ubuntu
operator system and software development programs such as a ``terminal window'' and the ``GVim text editor''. You will
also continue to practice using Slack to support communication with the teaching assistants and the course instructor.
Next, you will learn how to implement a Java program and to write a Markdown document, also discovering how to use the
course's automated grading tool to assess your progress towards correctly completing the project.

\section*{Suggestions for Success}

\begin{itemize}
  \setlength{\itemsep}{0pt}

\item {\bf Use the laboratory computers}. The computers in this laboratory feature specialized software for completing
  this course's laboratory and practical assignments. If it is necessary for you to work on a different machine, be sure
  to regularly transfer your work to a laboratory machine so that you can check its correctness. If you cannot use a
  laboratory computer and you need help with the configuration of your own laptop, then please carefully explain its
  setup to a teaching assistant or the course instructor when you are asking questions.

\item {\bf Follow each step carefully}. Slowly read each sentence in the assignment sheet, making sure that you
  precisely follow each instruction. Take notes about each step that you attempt, recording your questions and ideas and
  the challenges that you faced. If you are stuck, then please tell a teaching assistant or instructor what assignment
  step you recently completed.

\item {\bf Regularly ask and answer questions}. Please log into Slack at the start of a laboratory or practical session
  and then join the appropriate channel. If you have a question about one of the steps in an assignment, then you can
  post it to the designated channel. Or, you can ask a student sitting next to you or talk with a teaching assistant or
  the course instructor.

\item {\bf Store your files in GitHub}. Starting with this laboratory assignment, you will be responsible for storing
  all of your files (e.g., Java source code and Markdown-based writing) in a Git repository using GitHub Classroom.
  Please verify that you have saved your source code in your Git repository by using the ``{\tt git status}'' to ensure
  that everything is updated. You can see if your assignment submission meets the established correctness requirements
  by using the provided checking tools on your local computer and in checking the commits in GitHub.

\item {\bf Keep all of your files}. Don't delete your programs, output files, and written reports after you submit them
  through GitHub; you will need them again when you study for the quizzes and examinations and work on the other
  laboratory, practical, and final project assignments.

\item {\bf Back up your files regularly}. All of your files are regularly backed-up to the servers in the Department of
  Computer Science and, if you commit your files regularly, stored on GitHub. However, you may want to use a flash
  drive, Google Drive, or your favorite backup method to keep an extra copy of your files on reserve. In the event of
  any type of system failure, you are responsible for ensuring that you have access to a recent backup copy of all your
  files.

\item {\bf Explore teamwork and technologies}. While certain aspects of the laboratory assignments will be challenging
  to you, each part is designed to give you the opportunity to learn both fundamental concepts in the field of computer
  science and explore advanced technologies that are commonly deployed at a wide variety of companies. You should
  regularly communicate with your team and/or the teaching assistants and tutors to explore and develop new ideas.

\item {\bf Hone your technical writing skills}. Computer science assignments require to you write technical
  documentation and descriptions of your experiences when completing each task. Take extra care to ensure that your
  writing is interesting and both grammatically and technically correct, remembering that computer scientists must
  effectively communicate and collaborate with their team members and the tutors, teaching assistants, and instructor.

\item {\bf Review the Honor Code on the syllabus}. While you may discuss your assignments with others, copying source
  code or writing is a violation of the College's Honor Code.

\end{itemize}

\section*{Reading Assignment}

If you have not done so already, please read all of the relevant ``GitHub Guides'', available at
\url{https://guides.github.com/}, that explain how to use many of the features that GitHub provides. In particular,
please make sure that you have read guides such as ``Mastering Markdown'' and ``Documenting Your Projects on GitHub'';
each of them will help you to understand how to use both GitHub and GitHub Classroom. To do well on this assignment, you
should also read Chapters 1 and 2 in the course textbook, paying particularly close attention to Sections 1.5 and 2.1.
Please see the instructor or one of the teaching assistants if you have questions these reading assignments.

\section*{Using Your Computer Science Account}

In advance of today's lab you have already received the details about your Alden Hall computer account and learned how
to log on and use programs. You should ensure that you have recorded how to complete these steps in your notebook;
please report any problems as soon as they occur. You may use this account on any computer in Alden labs 101, 103, or
109. Your files are stored on a central server; you don't have to use the same machine every time you log on in a
laboratory.

Hours of lab availability are posted on the bulletin board in each lab and on the following Web site:
\url{http://www.cs.allegheny.edu/}; the on-duty lab monitor is always available in Alden 101.

\section*{Configuring Git and GitHub}

During this laboratory assignment and the subsequent laboratory and practical assignments, we will securely communicate
with the GitHub servers that will host all of the project templates and your submitted deliverables. In this assignment,
you will perform all of the steps to configure your account on GitHub and you will start your first assignment using
GitHub Classroom. Throughout this assignment, you should refer to the following web sites for more information:
\url{https://guides.github.com/} and, in particular, \url{https://guides.github.com/activities/hello-world/}. You can
also learn more about GitHub Classroom by visiting \url{https://classroom.github.com/}. As you will be required to use
Git and GitHub in all of the remaining laboratory and practical assignments and during the class sessions, please be
sure to keep a record of all of the steps that you complete and the challenges that you face. Please see the course
instructor or one of the teaching assistants if you are not able to complete a certain step or if you are not sure how
to proceed.

\begin{enumerate}

  \item If you do not already have a GitHub account, then please go to the GitHub web site and create one, making sure
    that you use your {\tt allegheny.edu} email address so that you can join the GitHub as a student at an accredited
    educational institution. You are also encouraged to sign up for GitHub's ``Student Developer Pack'' at
    \url{https://education.github.com/pack}, qualifying you to receive free software development tools. Also, please
    make sure that you add a description of yourself and an appropriate professional photograph to your GitHub profile.
    Unless your username is taken, you should also pick your GitHub username to be the same as your Google account. Now,
    in the {\tt \#labs} channel of our Slack team, please carefully type your full name, {\tt allegheny.edu} email
    address, and your new GitHub username.

  \item If you have never done so before, you must use the {\tt ssh-keygen} program to create secure-shell keys that you
    can use to support your communication with the GitHub servers. But, to start, this task requires you to type
    commands in a program that is known as a terminal. To run it, on the left side of your screen, click on the icon
    that contains the ``{\tt >}'' symbol. Alternatively, you can type the ``Super'' key, start typing the word
    ``terminal'', and then select that program. Another way to open a terminal involves typing the key combination {\tt
    <Ctrl>-<Alt>-t}. Once the terminal starts on Ubuntu, it will display as a box into which you can type commands.

  \item If you have not done so already, you will now need to run the {\tt ssh-keygen} command in your terminal window.
    Follow the prompts to create your keys and save them in the default directory. That is, you should press ``Enter''
    after you are prompted ``{\tt Enter file in which to save the key ...  :}'' and then type your selected passphrase
    whenever you are prompted to do so. Please note that a ``passphrase'' is like a password that you will type when you
    need to prove your identify to GitHub. What files does {\tt ssh-keygen} produce? Where does this program store these
    files by default? Do you have any questions about this step?

  \item Once you have created your ssh keys, you should raise your hand to invite either a teaching assistant or the
    course instructor to help you with the next steps. First, you must log into GitHub and look in the right corner for
    an account avatar with a down arrow. Click on this link and then select the ``Settings'' option. Now, scroll down
    until you find the ``SSH and GPG keys'' label on the left, click create a new ``SSH key'', and then upload your ssh
    key to GitHub. You can copy your to SSH key to the clipboard by going to the terminal and typing ``{\tt cat
    \textasciitilde{}/.ssh/id\_rsa.pub}'' command and then highlighting this output. When you are completing this step
    in your terminal window, please make sure that you only highlight the letters and numbers in your key---if you
    highlight any extra symbols or spaces then this step may not work correctly. Then, paste this into the text field in
    your web browser.

  \item Again, when you are completing these steps, please make sure that you take careful notes about the inputs,
    outputs, and behavior of each command. If there is something that you do not understand, then please ask the course
    instructor or the teaching assistant about it.

  \item Since this is your first laboratory assignment and you are still learning how to use the appropriate software,
    don't become frustrated if you make a mistake. Instead, use your mistakes as an opportunity for learning both about
    the necessary technology and the background and expertise of the other students in the class, the teaching
    assistants, and the course instructor. Remember, you can use Slack to talk with the instructor by using ``{\tt
    @gkapfham}'' in a channel.

\end{enumerate}

\section*{Accessing the Laboratory Assignment}

To access the laboratory assignment, you should go into the {\tt \#announcements} channel in our Slack team and find the
announcement that provides a link for it. Copy this link and paste it into your web browser. Now, you should accept the
laboratory assignment and see that GitHub Classroom created a new GitHub repository for you to access the assignments
starting materials and to store the completed version of your assignment. Specifically, to access your new GitHub
repository for this assignment, please click the green ``Accept'' button and then click the link that is prefaced with
the label ``Your assignment has been created here''. If you accepted the assignment and correctly followed these steps,
you should have created a GitHub repository with a name like
``Allegheny-Computer-Science-111-Fall-2017/computer-science-111-fall-2017-lab-1-gkapfham''.

Now you are ready to download the starting materials to your laboratory computer. Click the ``Clone or download'' button
and, after ensuring that you have selected ``Clone with SSH'', please copy this command to your clipboard. At this
point, you can open a new terminal window and type the command ``{\tt mkdir cs111F2017}''. To enter into this directory
you should now type ``{\tt cd cs111F2017}''. Next, you can type the command ``{\tt ls}'' and see that there are no files
or directories inside of this directory. By typing ``{\tt git clone}'' in your terminal and then pasting in the string
that you copied from the GitHub site you will download all of the code for this assignment.

In order to create a program, you need a text editor. There are many different text editors on your workstation and
you should feel free to explore these on your own. Since it is a powerful text editor known for helping computer
scientists ``edit text at the speed of thought'', in this class we will use the text editor called ``{\tt gvim}''.
Today, you will write your first program in {\tt gvim}.

There are several ways to launch {\tt gvim}, but for today please use a method described in this paragraph.  On the
left side of your screen, click on the icon that contains the ``{\tt >}'' symbol.  Alternatively, you can type the
``Super'' key, start typing the word ``terminal'', and then select that program.  Another way to open a terminal
involves typing the key combination {\tt <Ctrl>-<Alt>-t}.

% (If you don't see it, right-click on the desktop and choose ``Open in Terminal'' from the menu.)  This opens a {\em
%   terminal} window, a plain window with text. For today, you'll work with the terminal window rather than with folders,
% icons, and the mouse.

Now, you should type the following commands---exactly as shown---into the terminal window.  (The ``{\tt 1}'' is the
digit ``one'', not the letter ``ell.'') All Linux, Java, and {\tt gvim} commands are case-sensitive, so be sure to
capitalize the file name ``{\tt Lab1.java}'' but nothing else.  Don't worry if you make a mistake---just ask the course
instructor or a teaching assistant for help and then try again.

\vspace*{-.1in}
\begin{verbatim}
       mkdir cs111S2017
       cd cs111S2017
       mkdir labs
       cd labs
       mkdir lab1
       cd lab1
       gvim Lab1.java
\end{verbatim}
\vspace*{-.1in}

Once you have finished typing these commands and you are sure that they worked correctly, please reflect on what these
steps accomplished and then make a few notes about the process. Once you think that you understand the process
completely, please turn to a person sitting near you and explain it verbally. After discussing these steps with your
neighbor, did you both arrive at the same understanding of their purpose? If you did not, then please talk with a
teaching assistant.

After typing ``{\tt gvim Lab1.java}'', a new window should appear. This is the {\tt gvim} editor.  Since {\tt gvim} is a
modal editor, you may notice that if you start typing, nothing appears (unless you happen to hit certain letters such as
``{\tt i},'' ``{\tt o}'', ``{\tt a}'', and a few others). This is because you are not in ``insert mode.'' To get into
insert mode, just type the letter ``{\tt i}'' (lower case). Once you do this, the window should look like the one in
Figure \ref{gvim-insert}. Note the word ``{\tt --INSERT--}'' in the lower left corner!

Type the program from Figure \ref{lab1prog} into the window, substituting your actual name for the words ``Your Name''
and including today's date in place of that which is listed on the fourth line. When you are finished, press the ESC
key located in the upper left corner of the keyboard. This should remove the word ``{\tt --INSERT--}'' from the bottom
of the screen and take you out of insert mode.

Use the ``File/Save'' command to save your program. Alternatively, if you would like to use the keyboard to save your
file, you can press ``:w'' when you are not in insert mode.

Leaving the {\tt gvim} window open, go back to your terminal window. Type the command ``{\tt javac Lab1.java}'' at the
terminal window's prompt---this is the ``compile'' step of writing a Java program that transforms the source code into a
form that can be run on your computer.

If you get any error messages, go back into {\tt gvim} and try to figure out what you mis-typed and fix it. Once you
have solved the problem, make a note of the error and the solution for resolving it. Re-save your program and then
re-compile it (i.e., re-run the ``{\tt javac}'' command). If you cannot get the program to compile correctly, then
please talk with a teaching assistant or the course instructor.

When all errors are eliminated, type ``{\tt java Lab1}'' in the terminal window---this is the ``execute'' step that will
run your program and produce the designated output.  You should see your name, today's date, the lab number, and two
more lines of text. Make sure there are spaces separating words in your output (did you forget to put a space inside the
quotation marks after your last name?). If not, then repair the program and re-compile and re-run it.  Once the program
runs, please reflect on this process.  What step did you find to be the most challenging? Why?

Using the ``File/Print'' menu item, print out your program directly from {\tt gvim}. Pick up your output at one of the
two printers in a lab in Alden Hall.  Please see the course instructor if you have trouble printing.  Sign your name at
the top of your printout---this is the pledge that the work you are handing in was done according to the policies in
Allegheny College's Honor Code.

\vspace*{-.15in}
\section*{Write Your Own Program}
\vspace*{-.05in}

Following the steps from the previous part of this laboratory assignment, create a program with a different name (e.g.,
``{\tt Lab1Part2.java}''). Note that the name in your ``{\tt public class}'' statement must exactly match the portion of
the file name preceding the ``{\tt .java}''.  For instance, you must have ``{\tt public class Lab1Part2}'' if the file
name is ``{\tt Lab1Part2.java}''.

Make sure that your program outputs something different than the program that you wrote previously---but still include
your name and the date as shown in the example. Experiment with creating output that includes quotations, art work, or
technical diagrams.  At minimum, this program should create more lines of output than your first one. Try to
purposefully include errors in your program by, for instance, omitting the ``{\tt ;}'', capitalizing something
incorrectly, misspelling a Java keyword, or making other mistakes. As you learn by trying new things, ask additional
questions of or give a status update to the course instructor and a teaching assistant.

Since this is our first laboratory assignment and you are still learning how to use the appropriate hardware and
software, don't become frustrated if you make a mistake. Instead, use your mistakes as an opportunity for learning both
about the necessary technology and the background and expertise of the other students in the class, the teaching
assistants, and the course instructor.

\section*{Summary of the Required Deliverables}

This assignment invites you to submit printed and signed versions of the following deliverables:

\vspace*{-.1in}
\begin{enumerate}
  \setlength{\itemsep}{0in}
\item A commentary on the meaning and purpose of all the commands you typed in the terminal.
\item A properly commented and formatted version of {\tt Lab1.java} and {\tt Lab1Part2.java}.
\item The output from running {\tt Lab1} in the terminal window.
\item The output from running {\tt Lab1Part2} in the terminal window.
\end{enumerate}
\vspace*{-.1in}

In adherence to the Honor Code, students should complete this assignment on an individual basis. While it is appropriate
for students in this class to have high-level conversations about the assignment, it is necessary to distinguish
carefully between the student who discusses the principles underlying a problem with others and the student who produces
assignments that are identical to, or merely variations on, someone else's work.  With the exception of the {\tt
Lab1.java} program that you created in the first part of the assignment, deliverables that are nearly identical to the
work of others will be taken as evidence of violating the \mbox{Honor Code}.

%Please see the course instructor if you
% have questions about this policy.

% \section*{Deliverable}
% Hand in your stapled printed programs (there should be at least two, but you might do several more if you want) before
% the due date and time.

\end{document}
