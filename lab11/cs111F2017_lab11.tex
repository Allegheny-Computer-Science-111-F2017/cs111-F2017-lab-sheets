\documentclass[11pt]{article}

% NOTE: The "Edit" sections are changed for each assignment

% Edit these commands for each assignment

\newcommand{\assignmentduedate}{December 14}
\newcommand{\assignmentassignedate}{November 16}
\newcommand{\assignmentnumber}{Eleven}

\newcommand{\labyear}{2017}
\newcommand{\labday}{Thursday}
\newcommand{\labtime}{2:30 pm}
\newcommand{\labduetime}{5:00 pm}

\newcommand{\assigneddate}{Assigned: \labday, \assignmentassignedate, \labyear{} at \labtime{}}
\newcommand{\duedate}{Due: \labday, \assignmentduedate, \labyear{} at \labduetime{}}

% Edit these commands to give the name to the main program

\newcommand{\mainprogram}{\lstinline{MandelbrotBlackAndWhite}}
\newcommand{\mainprogramsource}{\lstinline{src/main/java/labten/MandelbrotBlackAndWhite.java}}

\newcommand{\secondprogram}{\lstinline{MandelbrotColor}}
\newcommand{\secondprogramsource}{\lstinline{src/main/java/labten/MandelbrotColor.java}}

% Edit this commands to describe key deliverables

\newcommand{\reflection}{\lstinline{writing/reflection.md}}
\newcommand{\timefile}{\lstinline{writing/time.md}}
\newcommand{\colorfile}{\lstinline{writing/color.md}}
\newcommand{\readme}{\lstinline{README.md}}

% Commands to describe key development tasks

% --> Running gatorgrader.sh
\newcommand{\gatorgraderstart}{\command{./gatorgrader.sh --start}}
\newcommand{\gatorgradercheck}{\command{./gatorgrader.sh --check}}

% --> Compiling and running program with gradle
\newcommand{\gradlebuild}{\command{gradle build}}
\newcommand{\gradlerun}{\command{gradle run}}

% Commands to describe key git tasks

\newcommand{\gitcommitfile}[1]{\command{git commit #1}}
\newcommand{\gitaddfile}[1]{\command{git add #1}}

\newcommand{\gitadd}{\command{git add}}
\newcommand{\gitcommit}{\command{git commit}}
\newcommand{\gitpush}{\command{git push}}
\newcommand{\gitpull}{\command{git pull}}

\newcommand{\gitaddmainprogram}{\command{git add src/main/java/labten/MandelbrotColorManager.java}}
\newcommand{\gitcommitmainprogram}{\command{git commit src/main/java/labten/MandelbrotColorManager.java -m "Your
descriptive commit message"}}

% Use this when displaying a new command

\newcommand{\command}[1]{``\lstinline{#1}''}
\newcommand{\program}[1]{\lstinline{#1}}
\newcommand{\url}[1]{\lstinline{#1}}
\newcommand{\channel}[1]{\lstinline{#1}}
\newcommand{\option}[1]{``{#1}''}
\newcommand{\step}[1]{``{#1}''}

\usepackage{pifont}
\newcommand{\checkmark}{\ding{51}}
\newcommand{\naughtmark}{\ding{55}}

\usepackage{listings}
\lstset{
  basicstyle=\small\ttfamily,
  columns=flexible,
  breaklines=true
}

\usepackage{fancyhdr}
\usepackage{fancyvrb}

\usepackage[margin=1in]{geometry}
\usepackage{fancyhdr}

\pagestyle{fancy}

\fancyhf{}
\rhead{Computer Science 111}
\lhead{Final Project}
\rfoot{Page \thepage}
\lfoot{\duedate}

\usepackage{titlesec}
\titlespacing\section{0pt}{6pt plus 4pt minus 2pt}{4pt plus 2pt minus 2pt}

\newcommand{\labtitle}[1]
{
  \begin{center}
    \begin{center}
      \bf
      CMPSC 111\\Introduction to Computer Science I\\
      Fall 2017\\
      \medskip
    \end{center}
    \bf
    #1
  \end{center}
}

\begin{document}

\thispagestyle{empty}

\labtitle{Final Project \\ \assigneddate{} \\ \duedate{}}

\section*{Introduction}

Throughout the semester, you have explored the fundamentals of computer science and Java programming by studying, in a
hands-on fashion, topics such as data and expressions, the use and creation of Java classes, conditionals and loops, and
arrays and lists. This final project invites you to explore, in greater detail, a real-world application of computer
science. You will learn more about how to use, implement, test, and evaluate different types of real-world computer
software. Since you will complete the final project with a partner, you will also learn more about how GitHub and the
Git version control system can effectively support collaborative software development.

Your project should result in a detailed report that includes all of your source code, in addition to written materials
and technical diagrams that highlight the key contributions of your work. This technical report should include a
description of why the chosen topic is important and discuss the implementation and/or experimentation that you
undertook. The written material should be precise, formal, appropriately formatted, grammatically correct, informative,
and interesting. The source code that you write must be carefully documented and tested. If you install and use existing
computer software (e.g., a Java library for creating computer graphics), the steps for installation and use should be
clearly documented in your report. Also, the report must explain the steps to run your own Java program. Finally, your
report must detail the work completed by each member of your partnership; individual contributions should also be
reflected in the Git repository's log. In addition to writing the aforementioned final report in Markdown, you will also
use Markdown to write and submit a project proposal and a status updated at the stated intermediate deadlines.

\section*{Suggestions for Success}

\begin{itemize}
  \setlength{\itemsep}{0pt}

\item {\bf Use the laboratory computers}. The computers in this laboratory feature specialized software for completing
  this course's laboratory and practical assignments. If it is necessary for you to work on a different machine, be sure
  to regularly transfer your work to a laboratory machine so that you can check its correctness. If you cannot use a
  laboratory computer and you need help with the configuration of your own laptop, then please carefully explain its
  setup to a teaching assistant or the course instructor when you are asking questions.

\item {\bf Follow each step carefully}. Slowly read each sentence in the assignment sheet, making sure that you
  precisely follow each instruction. Take notes about each step that you attempt, recording your questions and ideas and
  the challenges that you faced. If you are stuck, then please tell a teaching assistant or instructor what assignment
  step you recently completed.

\item {\bf Regularly ask and answer questions}. Please log into Slack at the start of a laboratory or practical session
  and then join the appropriate channel. If you have a question about one of the steps in an assignment, then you can
  post it to the designated channel. Or, you can ask a student sitting next to you or talk with a teaching assistant or
  the course instructor.

\item {\bf Store your files in GitHub}. As in the past laboratory assignments, you will be responsible for storing all
  of your files (e.g., Java source code and Markdown-based writing) in a Git repository using GitHub Classroom. Please
  verify that you have saved your source code in your Git repository by using \command{git status} to ensure that
  everything is updated. You can see if your assignment submission meets the established correctness requirements by
  using the provided checking tools on your local computer and by checking the commits in GitHub.

\item {\bf Keep all of your files}. Don't delete your programs, output files, and written reports after you submit them
  through GitHub; you will need them again when you study for the quizzes and examinations and work on the other
  laboratory, practical, and final project assignments.

\item {\bf Back up your files regularly}. All of your files are regularly backed-up to the servers in the Department of
  Computer Science and, if you commit your files regularly, stored on GitHub. However, you may want to use a flash
  drive, Google Drive, or your favorite backup method to keep an extra copy of your files on reserve. In the event of
  any type of system failure, you are responsible for ensuring that you have access to a recent backup copy of all your
  files.

\item {\bf Explore teamwork and technologies}. While certain aspects of the laboratory assignments will be challenging
  for you, each part is designed to give you the opportunity to learn both fundamental concepts in the field of computer
  science and explore advanced technologies that are commonly employed at a wide variety of companies. To explore and
  develop new ideas, you should regularly communicate with your team and/or the teaching assistants and tutors.

\item {\bf Hone your technical writing skills}. Computer science assignments require to you write technical
  documentation and descriptions of your experiences when completing each task. Take extra care to ensure that your
  writing is interesting and both grammatically and technically correct, remembering that computer scientists must
  effectively communicate and collaborate with their team members and the tutors, teaching assistants, and course
  instructor.

\item {\bf Review all of your past laboratory and practical assignments}. Now that you have completed many prior
  assignments, please review all of your prior work to ensure that you understand the concepts needed to explore
  real-world applications of computer science.

\item {\bf Review the Honor Code on the syllabus}. While you may discuss your assignments with others, copying source
  code or writing is a violation of Allegheny College's Honor Code.

\end{itemize}

\section*{Reading Assignment}

To ensure that you are best prepared to complete this final project, please review all of the chapters that we have
covered up to the release date of this assignment. As we cover new material during the remainder of the semester (e.g.,
arrays and exception handling), you are also encouraged to review that content as it may better enable you to complete a
high-quality final project.

\section*{Accessing the Final Project Assignment on GitHub}

To access the laboratory assignment, you should go into the \channel{\#announcements} channel in our Slack team and find
the announcement that provides a link for it. Now, make sure that the leader of your team also notes your team number
and first copies this link and pastes it into their web browser. Next, the team leader (i.e., the person you elect to
complete this step) will create their team with the name \command{Computer-Science-111-Fall-2017-Lab-11-Group-<group
label>} and then accept the laboratory assignment and see that GitHub Classroom created a new GitHub repository for your
team to access the assignment's starting materials and to store the completed version of your assignment. Note that the
team leader will have to type their group and lab details into a text field. For instance, if the team leader was in the
second group then that person would type ``Group 2 for Lab 11'' into the text field. At this point, each additional
member of the team can accept the assignment through GitHub. Please make sure that each of your team members joins the
team to which the instructor assigned them to work. Unless you provide the instructor with documentation of the
extenuating circumstances that you are facing, not accepting the assignment means that you automatically receive a
failing grade for it. Please see the course instructor with any questions.

Before you move to the next step of this assignment, please make sure that you read all of the content on the web site
for your new GitHub repository, paying close attention to the technical details about the commands that you will type
and the output that your program must produce. Now you are ready to download the starting materials to your laboratory
computer. Click the ``Clone or download'' button and, after ensuring that you have selected ``Clone with SSH'', please
copy this command to your clipboard. To enter the correct directory you should now type \command{cd cs111F2017}. Next,
you can type the command \command{ls} and see that there are some files or directories inside of this directory. By
typing \command{git clone} in your terminal and then pasting in the string that you copied from the GitHub site you will
download all of the code for this assignment. For the previous example, a student would run a \command{git clone}
command in the terminal in this fashion:

\begin{lstlisting}
  git clone git@github.com:Allegheny-Computer-Science-111-F2017/computer-science-111-fall-2017-lab-11-group-2-for-lab-11.git
\end{lstlisting}

After this command finishes, you can use \command{cd} to change into the new directory. If you want to \step{go back}
one directory from your current location, then you can type the command \command{cd ..}. Please continue to use the
\command{cd} and \command{ls} commands to explore the files that you automatically downloaded from GitHub. What files
and directories do you see? What do you think is their purpose? Spend some time exploring, sharing your discoveries with
your partner and a \mbox{teaching assistant}.

\section*{Understanding and Empirically Evaluating a Fractal Generator}

After developing a basic understanding of the \mainprogramsource{} and \secondprogramsource{} files, you should try to
compile and run these programs. To run the version of the program that produces a black-and-white image, you can type
the following command in your terminal window: \command{gradle -b build-bw.gradle run}. Alternatively, you can run the
program version that produces a color image by running the command \command{gradle -b build-color.gradle run}. Please
notice that these programs do not produce any output in the terminal window. Instead, you should see that they create a
graphics file in the same directory from where you run it. For instance, if you run the \secondprogram{} program you
will see that it creates a \command{mandelbrot-color.png} file. If you want to view the contents of this file, then you
can type the command \command{xdg-open mandelbrot-color.png} in your terminal window. What do you now see on the screen?
What are the characteristics of this image? How did the program create it?

Please use {\tt gvim} to display and edit the source code of the {\tt MandelbrotColor.java} file, noticing that it
declares an {\tt int} variable called {\tt max} and initializes it to the value of $1000$. How does the value of this
variable influence the time taken to create the Mandelbrot graphic? To answer this question, you can change the value of
max to, for instance, $100$, and rebuild and preface the \program{gradle} command with ``{\tt /usr/bin/time}'' in your
terminal. After running this command, the number that is postfixed with the ``{\tt user}'' label will give the amount of
time needed to create the fractal. As part of this assignment, you should set the {\tt max} variable to take on the
values of $10, 100, 1000, \mbox{and}, \- 10000$, timing the execution with {\tt /usr/bin/time} and using {\tt xdg-open}
to see how this changes the resulting visualization. Along with recording the execution time for each program
configuration, you should save each distinct graphic with a unique name and upload it to your GitHub repository.

\noindent
Now, find the line in the \secondprogramsource{} file that is written as:

\vspace*{.5em}
{\tt colors[i] = Color.HSBtoRGB(i/256f, 1, i/(i+8f));}
\vspace*{.5em}

\noindent This line of code configures the way in which \secondprogram{} will display the colors in the fractal. What
happens when you change the value of ``{\tt 256f}'' to a different floating point value? Will changing this value
influence the efficiency of the fractal generator? To answer these questions you should set this value to the values of
$32, 64, 128, 256, 512, \mbox{and}, 1024$---while keeping the value of {\tt max} set to the default of $1000$---and then
again use \command{/usr/bin/time} to measure the program's performance. In addition to recording the running time for
each program configuration, you should save each graphic with a unique name in your repository. As you vary both this
value and the value of the {\tt max} variable, you should determine how execution time varies and how the image changes.

It is worth making several additional points to ensure that you successfully complete this laboratory assignment. First,
you and your partner should alternate taking turns as you complete the work. For instance, you can share the task of
running the Java program and then saving its output to a file with a unique name. You can complete this task by first
using \program{gradle} and then using the \program{cp} command that accepts as a parameter a source and destination file
name so as to copy the source file to the destination. Please save all of your graphics files to the \program{graphics/}
directory. Then, another team member can help to create the data tables in Markdown. Specifically, when you record the
execution timings from running \secondprogram{}, you should record the numerical values in a Markdown-based table stored
in your version control repository. Finally, please add comments to the Java source codes that explain your intuition
about how the program works; your final submitted version of the Java files should be the one corresponding to their
last run.

% \noindent
% In conclusion, here are some points to remember when creating your program that manages a list:

% \begin{enumerate}
%   \setlength{\itemsep}{0pt}
% \item The provide source code is incomplete---make sure that you add all of the needed features.
% \item As in past assignments, your program only needs to have one {\tt main} method in one file.
% \item Don't forget that you and your team member(s) should all contribute evenly by typing in text and using commands,
%   like \command{git push} and \command{git pull}, to transfer files to and from GitHub.
% \end{enumerate}

\section*{Summary of the Required Deliverables}

\noindent Students do not need to submit printed source code or technical writing for any assignment in this course.
Instead, this assignment invites you to submit, using GitHub, the following deliverables.

\begin{enumerate}

  \setlength{\itemsep}{0in}

  \item Completed, fully commented, and properly formatted versions of the two source code files.
  \item The black-and-white version of the Mandelbrot graphic, saved with the default file name.
  \item Appropriately named files for all of the color versions of the Mandelbrot graphics.
  \item Data tables that show the performance timings associated with varying the two parameters.

  \item A written report, saved in the file \timefile{}, explaining the execution time trends in your Mandelbrot data
    tables. This Markdown-based document should furnish tables of executing timing numbers. Also, a written
    Markdown-based report, saved in the file \colorfile{}, detailing how the shape and color of the Mandelbrot graphic
    changes.

  \item Stored in \reflection{}, a three-paragraph reflection on the commands that you typed in \command{gvim} and the
    terminal window. This Markdown-based document should explain the input, output, and behavior of each command and the
    challenges that you confronted when using it. For every challenge that you encountered, please explain your solution
    for it. This document should also explain how your team collaborated to finish the assignment, with each team member
    writing their own paragraph inside of this Markdown file.

  \item A commit log in your GitHub repository that clearly shows that the team members effectively collaborated. That is,
    the commit log should show that commits were evenly made by all team members; the photograph of each member should
    appear in the commit log.

\end{enumerate}

\section*{Evaluation of Your Laboratory Assignment}

Using a report that the instructor shares with you through the commit log in GitHub, you will privately received a grade
on this assignment and feedback on your submitted deliverables. Your grade for the assignment will be a function of the
whether or not it was submitted in a timely fashion and if your repository contains correct versions of all of the
required Markdown and graphics files. Other factors will also influence your final grade on the assignment. The
instructor will also evaluate the accuracy of both your technical writing and the comments in your source code. Please
see the instructor if you have questions about the evaluation of this laboratory assignment.

All team members will receive the same baseline grade for the laboratory assignment. If there are extenuating
circumstances in which one or more of the team members do not effectively collaborate to complete this assignment, then
the course instructor will adjust the grade of specific team members so that it is higher or lower than the baseline
grade, as is fair and necessary. Please see the instructor if you do not understand how he assigns grades for
collaborative assignments. Finally, in adherence to the Honor Code, students should only complete this assignment with
their team members. Deliverables (e.g., Java source code or Markdown-based technical writing) that are nearly identical
to the work of outsiders of your team will be taken as evidence of violating the \mbox{Honor Code}. Please see the
course instructor with any questions about this assignment policy.

\end{document}
