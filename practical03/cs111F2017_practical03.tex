\documentclass[11pt]{article}

% NOTE: The "Edit" sections are changed for each assignment

% Edit these commands for each assignment

\newcommand{\assignmentduedate}{September 25}
\newcommand{\assignmentassignedate}{September 22}
\newcommand{\assignmentnumber}{Three}

\newcommand{\labyear}{2017}
\newcommand{\labday}{Monday}
\newcommand{\labtime}{9:00 am}

\newcommand{\assigneddate}{Assigned: \labday, \assignmentassignedate, \labyear{} at \labtime{}}
\newcommand{\duedate}{Due: \labday, \assignmentduedate, \labyear{} at \labtime{}}

% Edit these commands to give the name to the main program

\newcommand{\mainprogram}{\lstinline{DisplayComplementaryDrawingCanvas}}
\newcommand{\mainprogramsource}{\lstinline{src/main/java/practicalthree/DisplayComplementaryDrawingCanvas.java}}

\newcommand{\secondprogram}{\lstinline{PaintComplementaryDrawingCanvas}}
\newcommand{\secondprogramsource}{\lstinline{src/main/java/practicalthree/PaintComplementaryDrawingCanvas.java}}

% Edit this commands to describe key deliverables

\newcommand{\reflection}{\lstinline{writing/reflection.md}}

% Commands to describe key development tasks

% --> Running gatorgrader.sh
\newcommand{\gatorgraderstart}{\command{./gatorgrader.sh --start}}
\newcommand{\gatorgradercheck}{\command{./gatorgrader.sh --check}}

% --> Compiling and running program with gradle
\newcommand{\gradlebuild}{\command{gradle build}}
\newcommand{\gradlerun}{\command{gradle run}}

% Commands to describe key git tasks

% NOTE: Could be improved, problems due to nesting

\newcommand{\gitcommitfile}[1]{\command{git commit #1}}
\newcommand{\gitaddfile}[1]{\command{git add #1}}

\newcommand{\gitadd}{\command{git add}}
\newcommand{\gitcommit}{\command{git commit}}
\newcommand{\gitpush}{\command{git push}}
\newcommand{\gitpull}{\command{git pull}}

\newcommand{\gitcommitmainprogram}{\command{git commit src/main/java/practicalthree/DisplayComplementaryDrawingCanvas.java -m "Your
descriptive commit message"}}

% Use this when displaying a new command

\newcommand{\command}[1]{``\lstinline{#1}''}
\newcommand{\program}[1]{\lstinline{#1}}
\newcommand{\url}[1]{\lstinline{#1}}
\newcommand{\channel}[1]{\lstinline{#1}}
\newcommand{\option}[1]{``{#1}''}
\newcommand{\step}[1]{``{#1}''}

\usepackage{pifont}
\newcommand{\checkmark}{\ding{51}}
\newcommand{\naughtmark}{\ding{55}}

\usepackage{listings}
\lstset{
  basicstyle=\small\ttfamily,
  columns=flexible,
  breaklines=true
}

\usepackage{fancyhdr}

\usepackage[margin=1in]{geometry}
\usepackage{fancyhdr}

\pagestyle{fancy}

\fancyhf{}
\rhead{Computer Science 111}
\lhead{Practical Assignment \assignmentnumber{}}
\rfoot{Page \thepage}
\lfoot{\duedate}

\usepackage{titlesec}
\titlespacing\section{0pt}{6pt plus 4pt minus 2pt}{4pt plus 2pt minus 2pt}

\newcommand{\labtitle}[1]
{
  \begin{center}
    \begin{center}
      \bf
      CMPSC 111\\Introduction to Computer Science I\\
      Fall 2017\\
      \medskip
    \end{center}
    \bf
    #1
  \end{center}
}

\begin{document}

\thispagestyle{empty}

\labtitle{Practical \assignmentnumber{} \\ \assigneddate{} \\ \duedate{}}

\section*{Objectives}

To continue practicing the use of GitHub to access the files for a practical
assignment. Additionally, to practice using the Ubuntu operating system and
software development programs such as a ``terminal window'' and the ``GVim text
editor''. You will continue to practice using Slack to support communication
with the teaching assistants and the course instructor. Next, you will learn
how to fix a Java program, further discovering how the course's automated
grading tool assesses your progress towards correctly completing the project.
Finally, you will continue to learn more about computer graphics, variables,
and data types and the ways in which they are combined.

\section*{Suggestions for Success}

\begin{itemize}
  \setlength{\itemsep}{0pt}

\item {\bf Use the laboratory computers}. The computers in this laboratory feature specialized software for completing
  this course's Laboratory and practical assignments. If it is necessary for you to work on a different machine, be sure
  to regularly transfer your work to a laboratory machine so that you can check its correctness. If you cannot use a
  laboratory computer and you need help with the configuration of your own laptop, then please carefully explain its
  setup to a teaching assistant or the course instructor when you are asking questions.

\item {\bf Follow each step carefully}. Slowly read each sentence in the assignment sheet, making sure that you
  precisely follow each instruction. Take notes about each step that you attempt, recording your questions and ideas and
  the challenges that you faced. If you are stuck, then please tell a teaching assistant or instructor what assignment
  step you recently completed.

\item {\bf Regularly ask and answer questions}. Please log into Slack at the start of a laboratory or practical session
  and then join the appropriate channel. If you have a question about one of the steps in an assignment, then you can
  post it to the designated channel. Or, you can ask a student sitting next to you or talk with a teaching assistant or
  the course instructor.

\item {\bf Store your files in GitHub}. As in all of your past assignments, you will be responsible for storing
  all of your files (e.g., Java source code and Markdown-based writing) in a Git repository using GitHub Classroom.
  Please verify that you have saved your source code in your Git repository by using \command{git status} to ensure that
  everything is updated. You can see if your assignment submission meets the established correctness requirements by
  using the provided checking tools on your local computer and in checking the commits in GitHub.

\item {\bf Keep all of your files}. Don't delete your programs, output files, and written reports after you submit them
  through GitHub; you will need them again when you study for the quizzes and examinations and work on the other
  laboratory, practical, and final project assignments.

\item {\bf Back up your files regularly}. All of your files are regularly backed-up to the servers in the Department of
  Computer Science and, if you commit your files regularly, stored on GitHub. However, you may want to use a flash
  drive, Google Drive, or your favorite backup method to keep an extra copy of your files on reserve. In the event of
  any type of system failure, you are responsible for ensuring that you have access to a recent backup copy of all your
  files.

\item {\bf Explore teamwork and technologies}. While certain aspects of these
  assignments will be challenging for you, each part is designed to give you the
  opportunity to learn both fundamental concepts in the field of computer
  science and explore advanced technologies that are commonly employed at a wide
  variety of companies. To explore and develop new ideas, you should regularly
  communicate with your team and/or the teaching assistants and tutors.

\item {\bf Hone your technical writing skills}. Computer science assignments require to you write technical
  documentation and descriptions of your experiences when completing each task. Take extra care to ensure that your
  writing is interesting and both grammatically and technically correct, remembering that computer scientists must
  effectively communicate and collaborate with their team members and the tutors, teaching assistants, and course
  instructor.

\item {\bf Review the Honor Code on the syllabus}. While you may discuss your assignments with others, copying source
  code or writing is a violation of Allegheny College's Honor Code.

\end{itemize}

\section*{Reading Assignment}

If you have not done so already, please read all of the relevant ``GitHub Guides'', available at
\url{https://guides.github.com/}, that explain how to use many of the features that GitHub provides. In particular,
please make sure that you have read guides such as ``Mastering Markdown'' and ``Documenting Your Projects on GitHub'';
each of them will help you to understand how to use both GitHub and GitHub Classroom. Focusing on the content about
computer graphics, you should review Chapters 1 and 2 in the textbook, ensuring that you fully understand all of the
concepts that we discussed during class and investigated during prior practical and laboratory sessions. Please see the
instructor or one of the teaching assistants if you have questions these reading assignments.

\section*{Painting with Complementary Colors}

To access the practical assignment, you should go into the \channel{\#announcements} channel in our Slack team and find
the announcement that provides a link for it. Copy this link and paste it into your web browser. Now, you should accept
the practical assignment and see that GitHub Classroom created a new GitHub repository for you to access the
assignment's starting materials and to store the completed version of your assignment. Specifically, to access your new
GitHub repository for this assignment, please click the green ``Accept'' button and then click the link that is prefaced
with the label ``Your assignment has been created here''. If you accepted the assignment and correctly followed these
steps, you should have created a GitHub repository with a name like
``Allegheny-Computer-Science-111-Fall-2017/computer-science-111-fall-2017-practical-3-gkapfham''. Unless you provide the
instructor with documentation of the extenuating circumstances that you are facing, not accepting the assignment means
that you automatically receive a failing grade for it. Please follow the steps from the previous laboratory assignments
for finding your ``home base'' for this practical assignment; see the instructor if you are stuck on getting started.

Now, study the documentation in the provided source code to understand the type of output that your program should
produce. Of course, as you complete this practical assignment, make sure that you regularly commit your code to GitHub
and use descriptive messages that say what you fixed. Specifically, the purpose of this program is to create a canvas
that contains two colors in it. The first colored rectangle---which should take up exactly half of the page---should
display the color that was requested by the user. For instance, if the user inputs the RGB value of $(255,0,0)$, then
this region of the graphic should be filled with the color red. The second half of the image should be filled with the
``complement'' of the color requested by the user. For example, the complement of the red value of $255$ is the value
$255-255=0$ and the complement of the green value of $0$ is $255-0=255$. Intuitively, taking the complement of an RGB
should give you the color that is on the ``opposite'' side of the color wheel that you can view in the {\tt gimp}
program. So, what is the RGB value of the color red's complement? Does it appear on the opposite side of the color
wheel?

As a means of practicing the use of variables and expressions, and the creation of graphics in Java, you should make
some small additions to this program so that it fulfills its intended purpose. To accomplish this task, the first thing
that you must do is add a call to the {\tt page.fillRect} method so that it paints a rectangle of the color requested by
the user. Next, you should store in the variable called {\tt userComplementaryColor} a color that is the complement of
the one requested by the user. This means that you will have to write in the form of a Java expression the simple
equations that were intuitively explained in the previous paragraph. Now, the source code line ``{\tt
page.setColor(userComplementaryColor);}'' will cause the next shape to be filled with the complement of the color chosen
by the user. Finally, you will need to again call the {\tt page.fillRect} method with the parameters that will yield a
rectangle correctly drawn in the second half of the image. Please see the instructor or a teaching assistant if you have
questions about these steps.

After finishing the \mainprogramsource{} and \secondprogramsource{} files, you should repeatedly test you program to
make sure that it is creating the correct graphical output. For instance, if the user inputs through a file the RGB
value of $(255, 0, 0)$, then what are the colors in the left and right sides of the graphic? To test your program, you
may want to try several different RGB values that are listed in Figure 2.10 of your textbook. More advanced testing of
your program is possible by picking an RGB value from one side of the color wheel in {\tt gimp} and then checking that
your program correctly displays its complement (the one ``across'' the color wheel) in the graphic. Does your testing
reveal any limitations to our approach to calculating the complementary color?

Students who are interested in an additional challenge can investigate other ways to calculate complementary colors.
Another option for enterprising students is to further investigate the theory of color and actually implement ways to
display other types of colors derived from the one input by the user. For instance, you might try to display a color
that is ``analogous'' to the user's.

\section*{Checking the Correctness of Your Program and Writing}

As in the past assignments, you are provided with an automated tool for checking
the quality of your source code. Please note that the practical assignments do
not require you to produce a writing document as you do in the laboratory
assignments. However, to check your Java source code you can started with the
use of GatorGrader, type the command \gatorgraderstart{} in your terminal
window. Once this step completes you can type \gatorgradercheck{}. If your work
does not meet all of the assignment's requirements, then you will see the
following output in your terminal: \command{Overall, are there any mistakes in
the assignment? Yes}. If you do have mistakes in your assignment, then you will
need to review GatorGrader's output, find the mistake, and try to fix it.
Remember, this practical assignment provides you with Java source code that
purposefully contains mistakes --- your task is to find and fix these problems!

Once your program is building correctly, fulfilling at least some of the
assignment's requirements, you should transfer your files to GitHub using the
\gitcommit{} and \gitpush{} commands. For example, if you want to signal that
the \mainprogramsource{} file has been changed and is ready for transfer to
GitHub you would first type \gitcommitmainprogram{} in your terminal, followed
by typing \gitpush{} and checking to see that the transfer to GitHub is
successful. If you notice that transferring your code or writing to GitHub did
not work correctly, then please try to determine why, asking a teaching
assistant or the course instructor for help, if necessary.

After the course instructor enables \step{continuous integration} with a system
called Travis CI, when you use the \gitpush{} command to transfer your source
code to your GitHub repository, Travis CI will initialize a \step{build} of your
assignment, checking to see if it meets all of the requirements. If both your
source code and writing meet all of the established requirements, then you will
see a green \checkmark{} in the listing of commits in GitHub after awhile. If
your submission does not meet the requirements, a red \naughtmark{} will appear
instead. The instructor will reduce a student's grade for this assignment if the
red \naughtmark{} appears on the last commit in GitHub immediately before the
assignment's due date. Yet, if the green \checkmark{} appears on the last commit
in your GitHub repository, then you satisfied all of the main checks. Unless you
provide the course instructor with documentation of the extenuating
circumstances that you are facing, no late work will be considered towards your
completion grade for this practical assignment. You should aim to finish
practical assignments on the day that they are assigned; please see the
instructor if you do not understand this policy.

\subsection*{Reminder About the General Guidelines for Practical Sessions}

Since this is one of our first practical assignments and you are still learning how to use the Java programming
language, don't become frustrated if you make a mistake. Instead, use your mistakes as an opportunity for learning both
about the necessary technology and the background and expertise of the other students in the class, the teaching
assistants, and the course instructor.

\vspace*{-.05in}
\begin{itemize}

\itemsep 0in

\item {\bf Experiment}. Practical sessions are for learning by doing without the pressure of grades or ``right/wrong''
  answers. So try things! To learn about computer graphics, please intelligently experiment with the code, making small
  incremental changes and observing the output.

\item {\bf Practice Key Laboratory Skills.} As you are completing this assignment, practice using the {\tt gvim} text
  editor and the terminal until you can easily use their most important features.

\item {\bf Help One Another!} If your neighbor is struggling and you know what to do, offer your help. Don't ``do the
  work'' for them, but advise them on what to type or how to handle things. If you are stuck on a part of this practical
  assignment and you could not find any insights in either your textbook or online sources, formulate a question to ask
  your neighbor, a teaching assistant, or a course instructor. Try to strike the right balance between asking for help
  when you cannot solve a problem and working independently to find a solution.

\end{itemize}

\section*{Summary of the Required Deliverables}

\noindent Students do not need to submit printed source code or technical
writing for any assignment in this course. Instead, this assignment invites you
to submit, using GitHub, the following deliverables. Because this is a practical
assignment, you are not required to complete any technical writing.

\begin{enumerate}

\setlength{\itemsep}{0in}

\item A correct version of \mainprogramsource{} and \secondprogramsource{} that meets all of the established source code
  requirements and produces the desired graphical output.

\end{enumerate}

\section*{Evaluation of Your Practical Assignment}

Practical assignments are graded on a completion --- or ``checkmark'' --- basis.
If your GitHub repository has a \checkmark{} for the last commit before the
deadline then you will receive the highest possible grade for the assignment.
However, you will fail the assignment if you do not complete it correctly, as
evidenced by a red \naughtmark{} in your commit listing, by the set deadline for
completing the project. Please see the course instructor if you do not
understand how practical assignments are graded or you do not know how to
complete one of the specific tasks in this assignment.

% \section*{Adhering to the Honor Code}

% In adherence to the Honor Code, students should complete this assignment on an individual basis. While it is appropriate
% for students in this class to have high-level conversations about the assignment, it is necessary to distinguish
% carefully between the student who discusses the principles underlying a problem with others and the student who produces
% assignments that are identical to, or merely variations on, someone else's work. Deliverables (e.g., Java source code or
% Markdown-based technical writing) that are nearly identical to the work of others will be taken as evidence of violating
% the \mbox{Honor Code}. Please see the course instructor if you have questions about this policy.

\end{document}
