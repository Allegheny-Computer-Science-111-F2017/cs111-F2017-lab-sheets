\documentclass[11pt]{article}

% NOTE: The "Edit" sections are changed for each assignment

% Edit these commands for each assignment

\newcommand{\assignmentduedate}{November 20}
\newcommand{\assignmentassignedate}{November 13}
\newcommand{\assignmentnumber}{Two}

\newcommand{\labyear}{2017}
\newcommand{\assignedday}{Monday}
\newcommand{\dueday}{Monday}
\newcommand{\labtime}{9:00 am}

\newcommand{\assigneddate}{Announced: \assignedday, \assignmentassignedate, \labyear{} at \labtime{}}
\newcommand{\duedate}{Quiz: \dueday, \assignmentduedate, \labyear{} at \labtime{}}

% Edit these commands to give the name to the main program

\newcommand{\mainprogram}{\lstinline{DisplayOutput}}
\newcommand{\mainprogramsource}{\lstinline{src/main/java/labone/DisplayOutput.java}}

% Edit this commands to describe key deliverables

\newcommand{\reflection}{\lstinline{writing/reflection.md}}

% Commands to describe key development tasks

% --> Running gatorgrader.sh
\newcommand{\gatorgraderstart}{\command{./gatorgrader.sh --start}}
\newcommand{\gatorgradercheck}{\command{./gatorgrader.sh --check}}

% --> Compiling and running program with gradle
\newcommand{\gradlebuild}{\command{gradle build}}
\newcommand{\gradlerun}{\command{gradle run}}

% Commands to describe key git tasks

% NOTE: Could be improved, problems due to nesting

\newcommand{\gitcommitfile}[1]{\command{git commit #1}}
\newcommand{\gitaddfile}[1]{\command{git add #1}}

\newcommand{\gitadd}{\command{git add}}
\newcommand{\gitcommit}{\command{git commit}}
\newcommand{\gitpush}{\command{git push}}
\newcommand{\gitpull}{\command{git pull}}

\newcommand{\gitcommitmainprogram}{\command{git commit src/main/java/labone/DisplayOutput.java -m "Your
descriptive commit message"}}

% Use this when displaying a new command

\newcommand{\command}[1]{``\lstinline{#1}''}
\newcommand{\program}[1]{\lstinline{#1}}
\newcommand{\url}[1]{\lstinline{#1}}
\newcommand{\channel}[1]{\lstinline{#1}}
\newcommand{\option}[1]{``{#1}''}
\newcommand{\step}[1]{``{#1}''}

\usepackage{pifont}
\newcommand{\checkmark}{\ding{51}}
\newcommand{\naughtmark}{\ding{55}}

\usepackage{listings}
\lstset{
  basicstyle=\small\ttfamily,
  columns=flexible,
  breaklines=true
}

\usepackage{fancyhdr}

\usepackage[margin=1in]{geometry}
\usepackage{fancyhdr}

\pagestyle{fancy}

\fancyhf{}
\rhead{Computer Science 111}
\lhead{Quiz \assignmentnumber{}}
\rfoot{Page \thepage}
\lfoot{\duedate}

\usepackage{titlesec}
\titlespacing\section{0pt}{6pt plus 4pt minus 2pt}{4pt plus 2pt minus 2pt}

\newcommand{\guidetitle}[1]
{
  \begin{center}
    \begin{center}
      \bf
      CMPSC 111\\Introduction to Computer Science I\\
      Fall 2017\\
      \medskip
    \end{center}
    \bf
    #1
  \end{center}
}

\begin{document}

\thispagestyle{empty}

\guidetitle{Quiz \assignmentnumber{} Study Guide \\ \assigneddate{} \\ \duedate{}}

\section*{Introduction}

This course will have its second quiz on Monday, November 20, 2017 from 9:00 to 9:50 am. The quiz will be ``closed
notes'' and ``closed book'' and it will cover the following materials. Please review the ``Course Schedule'' on the Web
site for the course to see the content and slides that we have covered to this date. Students may post questions about
this material to our Slack team.

\begin{itemize}

  \itemsep -.01in

  \item All previous chapters in Lewis \& Loftus (e.g., ``Data and Expressions'')

  \item Chapter Five in Lewis \& Loftus, Sections 5.1--5.6 (i.e., ``Conditionals and Loops'')

  \item Chapter Six in Lewis \& Loftus, Sections 6.1--6.4 (i.e., ``More Conditionals and Loops'')

  \item Chapter Eight in Lewis \& Loftus, Sections 8.1--8.4 (i.e., ``Arrays'')

  % \item Chapter Eleven in Lewis \& Loftus, Sections 11.1--11.6 (i.e., ``Exceptions'')

  \item Using the basic commands in the Linux operating system; editing in {\tt gvim}, compiling and executing
    programs in Linux; knowledge of the basic commands for using {\tt git} and GitHub.

\end{itemize}

\vspace*{-.05in}
\noindent The quiz will be a mix of questions that have a form such as fill in the blank, short answer, true/false, and
completion.  The emphasis will be on the following topics:

\vspace*{-.05in}
\begin{itemize}

  \itemsep 0in

\item Fundamental concepts in computing and the Java language (e.g., definitions and background)

\item Practical laboratory techniques (e.g., editing, compiling, and running programs; effectively using files and
  directories; correctly using GitHub through the command-line {\tt git} program)

\item Understanding Java programs (e.g., given a short, perhaps even one line, source code segment written in Java,
  understand what it does and be able to precisely describe its output).

\item Composing Java statements and programs, given a description of what should be done. Students should be completely
  comfortable writing short source code statements that are in nearly-correct form as Java code. While your program may
  contain small syntactic errors, it is not acceptable to ``make up'' features of the Java programming language that do
  not exist in the language itself---so, please do not call a ``{\tt solveQuestionThree()}'' method!

\end{itemize}

\noindent No partial credit will be given for questions that are true/false, completion, or fill in the blank. Minimal
partial credit may be awarded for the questions that require a student to write a short answer. You are strongly
encouraged to write short, precise, and correct responses to all of the questions. When you are taking the quiz, you
should do so as a ``point maximizer'' who first responds to the questions that you are most likely to answer correctly
for full points. Please keep the time limitation in mind as you are absolutely required to submit the examination at the
end of the class period unless you have written permission for extra time from a member of the Learning Commons.
Students who do not submit their quiz on time will have their overall point total reduced. Finally, you must complete
the entire quiz while adhering to the Honor Code.

\section*{Reminder Concerning the Honor Code}

\noindent Students are required to fully adhere to the Honor Code during the completion of this quiz. More details about
the Allegheny College Honor Code are provided on the syllabus. Students are strongly encouraged to carefully review the
full statement of the Honor Code before taking this quiz.

% \noindent The following provides you with a review of Honor Code statement from the course syllabus:

% The Academic Honor Program that governs the entire academic program at Allegheny College is described in the Allegheny
% Academic Bulletin.  The Honor Program applies to all work that is submitted for academic credit or to meet non-credit
% requirements for graduation at Allegheny College.  This includes all work assigned for this class (e.g., examinations,
%   laboratory assignments, and the final project).  All students who have enrolled in the College will work under the Honor
% Program.  Each student who has matriculated at the College has acknowledged the following pledge:

% \vspace*{-.11in}
% \begin{quote}
%   I hereby recognize and pledge to fulfill my responsibilities, as defined in the Honor Code, and to maintain the
%   integrity of both myself and the College community as a whole.
% \end{quote}
% \vspace*{-.11in}

\section*{Detailed Review of Content}

The listing of topics in the following subsections is not exhaustive; rather, it serves to illustrate the types of
concepts that students should study as they prepare for the quiz. Please see the course instructor during office hours
if you have questions about any of the content listed in this section. Note that the material for Chapter Five was also
covered on the previous examination; material from this chapter will also be covered on this examination. The
examination will also assume that students are familiar with the material covered in the prior chapters (e.g., variables
and data types).

\subsection*{Chapter Five}

\begin{itemize}

  \itemsep -.015in
  \item The meaning and purpose of boolean expressions in conditional logic
  \item The different logical operators available for use in boolean expressions
  \item The overall structure and purpose of {\tt if} statements in Java
  \item How to use a truth table to understand the meaning of {\tt if} statements
  \item Best practices for comparing variables of different data types
  \item The meaning and purpose of looping constructs in the Java language
  \item The overall structure and purpose of {\tt while} statements in Java
  \item How {\tt break} and {\tt continue} statements work in looping constructs

\end{itemize}

\subsection*{Chapter Six}

\begin{itemize}

  \itemsep -.015in
  \item The meaning and purpose of {\tt switch} statements in the Java language
  \item The purpose of the {\tt default} case in a {\tt switch} statement
  \item The overall structure and purpose of {\tt do} statements in the Java language
  \item How {\tt do} statements are similar to and different from {\tt while} statements
  \item How to draw a technical diagram to represent a looping construct
  \item The meaning and purpose of {\tt for} loops in the Java programming language
  \item How to use a {\tt while} loop and an {\tt Iterator} to iterate through an {\tt ArrayList} instance
  \item Best practices for picking a specific looping construct for a problem requiring iteration
  \item How and when to use {\tt break} and {\tt continue} statements in a looping construct

\end{itemize}

\subsection*{Chapter Eight}

\begin{itemize}

  \itemsep -.015in
  \item An example of a problem that is best solved through the use of an array
  \item The types of technical diagrams that are best suited to visualizing an array
  \item The benefits and drawbacks associated with using arrays in a Java program
  \item The means by which you define and use arrays in the Java programming language
  \item The key characteristics of the array data structure (e.g., stores a single type of data)
  \item The meaning of the word ``index'' and how it connects to the array data structure
  \item The meaning and purpose of arrays bounds checking in the Java programming language
  \item How to combine {\tt for} loops and arrays in order to solve a problem
  \item How arrays are used to accept command-line arguments as input to a program
  \item An intuitive understanding of the meaning and purpose of two-dimensional arrays

\end{itemize}

\end{document}
