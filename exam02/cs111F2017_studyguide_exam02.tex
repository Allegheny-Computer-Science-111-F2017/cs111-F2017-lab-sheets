\documentclass[11pt]{article}

% NOTE: The "Edit" sections are changed for each assignment

% Edit these commands for each assignment

\newcommand{\assignmentduedate}{December 14}
\newcommand{\assignmentassignedate}{December 11}
\newcommand{\assignmentnumber}{Two}

\newcommand{\labyear}{2017}
\newcommand{\assignedday}{Monday}
\newcommand{\dueday}{Thursday}
\newcommand{\labtime}{9:00 am}

\newcommand{\assigneddate}{Announced: \assignedday, \assignmentassignedate, \labyear{} at \labtime{}}
\newcommand{\duedate}{Exam: \dueday, \assignmentduedate, \labyear{} at \labtime{}}

% Edit these commands to give the name to the main program

\newcommand{\mainprogram}{\lstinline{DisplayOutput}}
\newcommand{\mainprogramsource}{\lstinline{src/main/java/labone/DisplayOutput.java}}

% Edit this commands to describe key deliverables

\newcommand{\reflection}{\lstinline{writing/reflection.md}}

% Commands to describe key development tasks

% --> Running gatorgrader.sh
\newcommand{\gatorgraderstart}{\command{./gatorgrader.sh --start}}
\newcommand{\gatorgradercheck}{\command{./gatorgrader.sh --check}}

% --> Compiling and running program with gradle
\newcommand{\gradlebuild}{\command{gradle build}}
\newcommand{\gradlerun}{\command{gradle run}}

% Commands to describe key git tasks

% NOTE: Could be improved, problems due to nesting

\newcommand{\gitcommitfile}[1]{\command{git commit #1}}
\newcommand{\gitaddfile}[1]{\command{git add #1}}

\newcommand{\gitadd}{\command{git add}}
\newcommand{\gitcommit}{\command{git commit}}
\newcommand{\gitpush}{\command{git push}}
\newcommand{\gitpull}{\command{git pull}}

\newcommand{\gitcommitmainprogram}{\command{git commit src/main/java/labone/DisplayOutput.java -m "Your
descriptive commit message"}}

% Use this when displaying a new command

\newcommand{\command}[1]{``\lstinline{#1}''}
\newcommand{\program}[1]{\lstinline{#1}}
\newcommand{\url}[1]{\lstinline{#1}}
\newcommand{\channel}[1]{\lstinline{#1}}
\newcommand{\option}[1]{``{#1}''}
\newcommand{\step}[1]{``{#1}''}

\usepackage{pifont}
\newcommand{\checkmark}{\ding{51}}
\newcommand{\naughtmark}{\ding{55}}

\usepackage{listings}
\lstset{
  basicstyle=\small\ttfamily,
  columns=flexible,
  breaklines=true
}

\usepackage{fancyhdr}

\usepackage[margin=1in]{geometry}
\usepackage{fancyhdr}

\pagestyle{fancy}

\fancyhf{}
\rhead{Computer Science 111}
\lhead{Exam \assignmentnumber{}}
\rfoot{Page \thepage}
\lfoot{\duedate}

\usepackage{titlesec}
\titlespacing\section{0pt}{6pt plus 4pt minus 2pt}{4pt plus 2pt minus 2pt}

\newcommand{\guidetitle}[1]
{
  \begin{center}
    \begin{center}
      \bf
      CMPSC 111\\Introduction to Computer Science I\\
      Fall 2017\\
      \medskip
    \end{center}
    \bf
    #1
  \end{center}
}

\begin{document}

\guidetitle{Exam \assignmentnumber{} Study Guide \\ \assigneddate{} \\ \duedate{}}

\section*{Introduction}

This course will have its final exam on Friday, \assignmentduedate{}, 2017 from 9:00 am to 12:00 noon. The exam will be
``closed notes'' and ``closed book'' and it will cover the following materials. Please review the ``Course Schedule'' on
the Web site for the course to see the content and slides that we have covered to this date. Students may post questions
about this material to our Slack team. The instructor encourages students to form study groups to review for this
upcoming examination.

\begin{itemize}

  \itemsep 0in

  \item Chapter One in Lewis \& Loftus (i.e., ``Introduction to Computation and Programming'')

  \item Chapter Two in Lewis \& Loftus, Sections 2.1--2.9 (i.e., ``Data and Expressions'')

  \item Chapter Three in Lewis \& Loftus, Sections 3.1--3.7 (i.e., ``Using Classes and Objects'')

  \item Chapter Four in Lewis \& Loftus, Sections 4.1--4.9 (i.e., ``Writing Classes'')

  \item Chapter Five in Lewis \& Loftus, Sections 5.1--5.6 (i.e., ``Conditionals and Loops'')

  \item Chapter Six in Lewis \& Loftus, Sections 6.1--6.4 (i.e., ``More Conditionals and Loops'')

  \item Chapter Eight in Lewis \& Loftus, Sections 8.1--8.4 (i.e., ``Arrays'')

  \item Chapter Eleven in Lewis \& Loftus, Sections 11.1--11.6 (e.g., ``Exceptions'')

  \item Chapter Twelve in Lewis \& Loftus, Sections 12.1--12.3 (e.g., ``Recursion'')

  \item Using the many commands in the Linux operating system; editing in {\tt gvim}, compiling and executing
    programs in Linux; knowledge of the basic commands for using {\tt git} and GitHub.

  \item Your class notes and lecture slides, labs 1--10, practicals 1--10, and the handouts from lab.

\end{itemize}

\noindent Like the past quiz and exam, this exam will be a mix of questions that have a form such as fill in the
blank, short answer, true/false, and completion. The emphasis will be on the following topics:

\vspace*{-.05in}
\begin{itemize}

  \itemsep 0in

  \item Fundamental concepts in computing and the Java language (e.g., definitions and background).

  \item Fundamental concepts in programming languages (e.g., conditional logic and iteration).

  \item Advanced concepts in programming languages (e.g., exception handling and recursion).

  \item Practical laboratory techniques (e.g., editing, compiling, and running programs; effectively using files and
    directories; correctly using GitHub through the command-line {\tt git} program).

  \item Understanding Java programs (e.g., given a short, perhaps even one line, source code segment written in Java,
    understand what it does and be able to precisely describe its output).

  \item Composing Java statements and programs, given a description of what should be done. Students should be completely
    comfortable writing short source code statements that are in nearly correct form as Java code. While your program may
    contain small syntactic errors, it is not acceptable to ``make up'' features of the Java programming language that do
    not exist in the language itself---so, please do not call a ``{\tt solveQuestionThree()}'' method!

\end{itemize}

\noindent No partial credit will be given for questions that are true/false, completion, or fill in the blank. Minimal
partial credit may be awarded for the questions that require a student to write a short answer. You are strongly
encouraged to write short, precise, and correct responses to all of the questions. When you are taking the exam, you
should do so as a ``point maximizer'' who first responds to the questions that you are most likely to answer correctly
for full points. Please make sure that you review the past quiz so that you can comfortably answer its questions.
Students should keep the time limitation in mind as they are absolutely required to submit the examination at the end of
the class period unless they have written permission for extra time from a member of the Learning Commons. Students who
do not submit their exam on time will have their overall point total reduced. Please see the course instructor if you
have questions about these policies.

% \vspace*{-.15in}
\section*{Reminder Concerning the Honor Code}

\noindent Students are required to fully adhere to the Honor Code during the completion of this exam. More details about
the Allegheny College Honor Code are provided on the syllabus. Students are strongly encouraged to carefully review the
full statement of the Honor Code before taking this exam.

\noindent The following provides you with a review of Honor Code statement from the course syllabus:

The Academic Honor Program that governs the entire academic program at Allegheny College is described in the Allegheny
Academic Bulletin. The Honor Program applies to all work that is submitted for academic credit or to meet non-credit
requirements for graduation at Allegheny College. This includes all work assigned for this class (e.g., examinations,
laboratory assignments, and the final project). All students who have enrolled in the College will work under the Honor
Program.  Each student who has matriculated at the College has acknowledged the following pledge:

\begin{quote}
  I hereby recognize and pledge to fulfill my responsibilities, as defined in the Honor Code, and to maintain the
  integrity of both myself and the College community as a whole.
\end{quote}

\noindent Students who have questions about Allegheny College's Honor Code and how it applies to the completion of a
quiz or an examination in Computer Science 111, should immediately schedule a meeting with the course instructor to
openly discuss their questions and concerns.

\section*{Detailed Review of Content}

The listing of topics in the following subsections is not exhaustive; rather, it serves to illustrate the types of
concepts that students should study as they prepare for the exam. Please see the instructor during office hours if you
have questions about any of the content listed in this section.

\subsection*{Chapter One}

\begin{itemize}

  \item Basic understanding of computer hardware and software
  \item Computer number systems (e.g., binary and decimal)
  \item Purpose for and steps of the fetch-decode-execute cycle in the CPU
  \item Layout of and access techniques for computer memory
  \item Knowledge of computer networking methods and programs
  \item Basic syntax and semantics of the Java programming language
  \item Input(s) and output(s) of the Java compiler and virtual machine
  \item Basic principles of the object-oriented programming paradigm

\end{itemize}

\subsection*{Chapter Two}

\begin{itemize}

  \item Using escape sequences in the output of Java programs
  \item Ways to perform input and output in a Java program
  \item The variety of data types available to Java programmers
  \item The declaration of and assignment of values to variables
  \item Operators and operator precedence in Java expressions
  \item Techniques for converting variables from one data type to another
  \item Computer graphics and related topics such as pixels and screen resolution
  \item The use of the RGB system for specifying colors in Java programs

\end{itemize}

\subsection*{Chapter Three}

\begin{itemize}

  \item The steps for creating a new instance of a Java class
  \item How to use technical diagrams to visualize an object in memory
  \item The meaning of the term ``alias'' in a Java program
  \item The creation and use of Strings in the Java programming language
  \item The ways in which Java packages promote high-quality programming
  \item The variety of ways in which you can create and use random numbers
  \item How to call and use the methods provided by the {\tt java.lang.Math} class
  \item Ways in which programs create formatted output in a terminal window

\end{itemize}

\subsection*{Chapter Four}

\begin{itemize}

  \item Best practices for organizing and structuring a Java class
  \item How to declare an instance variable and an instance method
  \item The ways in which encapsulation promotes good object-oriented design
  \item The visibility modifiers (e.g., {\tt private}) that support encapsulation in the Java language
  \item The meaning and purpose of accessor and mutator methods in Java programs
  \item All of the key parts of a Java method (e.g., parameters and {\tt return} statements)
  \item Appropriate strategies for implementing constructors in Java programs
  \item A basic understanding of the graphical objects seen in user interfaces

\end{itemize}

\subsection*{Chapter Five}

\begin{itemize}

  \item The meaning and purpose of boolean expressions in conditional logic
  \item The different logical operators available for use in boolean expressions
  \item The overall structure and purpose of {\tt if} statements in Java
  \item How to use a truth table to understand the meaning of {\tt if} statements
  \item Best practices for comparing variables of different data types
  \item The meaning and purpose of looping constructs in the Java language
  \item The overall structure and purpose of {\tt while} statements in Java
  \item How {\tt break} and {\tt continue} statements work in looping constructs
  \item The ways in which an {\tt Iterator} can be used in a Java program
  \item The ways in which an {\tt ArrayList} can be used in a Java program

\end{itemize}

\subsection*{Chapter Six}

\begin{itemize}

  \item The meaning and purpose of {\tt switch} statements in the Java language
  \item The purpose of the {\tt default} case in a {\tt switch} statement
  \item The overall structure and purpose of {\tt do} statements in the Java language
  \item How {\tt do} statements are similar to and different from {\tt while} statements
  \item How to draw a technical diagram to represent a looping construct
  \item The meaning and purpose of {\tt for} loops in the Java programming language
  \item Best practices for picking a specific looping construct for a problem requiring iteration
  \item How and when to use {\tt break} and {\tt continue} statements in a looping construct

\end{itemize}

\subsection*{Chapter Eight}

\begin{itemize}

  \item An example of a problem that is best solved through the use of an array
  \item The types of technical diagrams that are best suited to visualizing an array
  \item The benefits and drawbacks associated with using arrays in a Java program
  \item The means by which you define and use arrays in the Java programming language
  \item The key characteristics of the array data structure (e.g., stores a single type of data)
  \item The meaning of the word ``index'' and how it connects to the array data structure
  \item The meaning and purpose of arrays bounds checking in the Java programming language
  \item How to combine {\tt for} loops and arrays in order to solve a computational problem
  \item How arrays are used to accept command-line arguments as input to a program
  \item An intuitive understanding of the meaning and purpose of two-dimensional arrays

\end{itemize}

\subsection*{Chapter Eleven}

\begin{itemize}

  \item Concrete examples of when and why exceptions might be thrown in a Java program
  \item Strategies for dealing with exceptions that arise during use of a Java program
  \item An explanation for why division by zero will raise an exception in a Java program
  \item The similarities and differences between caught and uncaught exceptions
  \item The structure and purpose of the {\tt try-catch} construct in Java programs
  \item The structure and purpose of the {\tt finally} clause in a Java program
  \item The ways in which exceptions are ``propagated'' when running a Java program
  \item How the Java programming language supports the handling of input/output exceptions
  \item Concrete examples of at least three different exception classes available in Java

\end{itemize}

\subsection*{Chapter Twelve}

\begin{itemize}

  \item The ways in which recursion helps a program to achieve repetition
  \item How recursion and iteration and similar to and different from each other
  \item The ability to trace the execution of a recursive method using a call tree
  \item A basic understanding of how the program stack supports recursion
  \item The ability to identify the base and recursive cases in a method
  \item The circumstances in which a method would enter an infinite recursion
  \item Some applications of recursive methods (e.g., the factorial and GCD functions)
  \item How recursion can be used to solve advanced problems such as maze navigation

\end{itemize}

\end{document}
